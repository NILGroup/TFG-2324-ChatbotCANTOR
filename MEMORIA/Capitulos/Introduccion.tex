\chapter{Introducción}
\label{cap:introduccion}


\chapterquote{Hay enfermos incurables, pero ninguno incuidable}{Francesc Torralba}


El avance en el campo de la medicina en el último siglo ha permitido un notable aumento en la esperanza de vida a nivel mundial. Sin embargo, este alargamiento de la vida también ha traído consigo un aumento en las enfermedades relacionadas con la vejez, como el Alzheimer. A medida que vivimos más tiempo, enfrentamos un mayor riesgo de desarrollar estas condiciones.


\section{Motivación}

A pesar de los avances en la investigación médica, todavía no se ha encontrado una cura definitiva para el Alzheimer. La mayoría de los tratamientos se centran en aliviar los síntomas y ralentizar la progresión de la enfermedad. Entre estos tratamientos, la terapia de reminiscencia ha surgido como una opción no farmacológica destacada.

La terapia de reminiscencia se centra en estimular los recuerdos del pasado del paciente, lo que puede tener beneficios significativos en su bienestar social, mental y emocional. Ayuda a las personas a recordar y compartir experiencias pasadas, lo que puede ser reconfortante y estimulante, especialmente para aquellos que enfrentan el desafío de la pérdida de memoria asociada con el Alzheimer.

Durante el período académico 2023-2024, se desarrolló la aplicación YayoBot con el objetivo de facilitar a los terapeutas la realización de terapias basadas en reminiscencia, simplificando y agilizando el proceso. Este desarrollo parte desde cero y busca crear un chatbot funcional y útil, capaz de mantener una conversación útil con el paciente para extraer la información necesaria.

Este Trabajo de Fin de Grado tiene como propósito asistir a terapeutas, familiares o amigos de pacientes con demencia en la obtención del material necesario para llevar a cabo terapias de reminiscencia. Se busca mejorar la eficacia de estas terapias y, en consecuencia, la calidad de vida tanto de los pacientes como de sus familiares. 
\subsection{Enfermedad de Alzheimer}
El Alzheimer, una enfermedad neurodegenerativa progresiva y devastadora, afecta a millones de personas en todo el mundo. Se caracteriza por la pérdida gradual de la memoria y otras funciones cognitivas, lo que eventualmente conduce a la incapacidad para llevar a cabo las actividades diarias más básicas. Su curso clínico se divide en varias etapas distintas, cada una con sus propias características y desafíos.
\begin{itemize}
	\item Etapa Temprana o Leve:\\
	En esta fase inicial, los síntomas pueden pasar desapercibidos o atribuirse a simples descuidos. La persona afectada puede experimentar dificultades para recordar nombres, eventos recientes o encontrar las palabras adecuadas en conversaciones. A pesar de estos desafíos, generalmente conservan la capacidad de realizar tareas cotidianas con cierta independencia. Sin embargo, es posible que comiencen a perder interés en actividades previamente disfrutadas.
	
	\item Etapa Intermedia o Moderada:\\
	A medida que la enfermedad progresa, los síntomas se vuelven más evidentes y problemáticos. La pérdida de memoria se vuelve más pronunciada, con dificultades para reconocer a familiares y amigos cercanos. Además, pueden surgir problemas de orientación en tiempo y espacio, lo que puede resultar en desorientación incluso en entornos familiares. Las habilidades de comunicación también se ven afectadas, con dificultades para seguir conversaciones o expresar pensamientos de manera coherente.
	
	\item Etapa Avanzada o Severa:\\
	En esta etapa tardía, el Alzheimer alcanza su punto más devastador. La pérdida de memoria es profunda y completa, con una incapacidad para recordar incluso eventos recientes o reconocer caras familiares. La persona afectada puede experimentar cambios significativos en la personalidad y el comportamiento, volviéndose agitada, ansiosa o incluso agresiva en ocasiones. La capacidad para realizar actividades básicas de la vida diaria, como vestirse o alimentarse, se ve seriamente comprometida, y la supervisión constante se vuelve esencial.
\end{itemize}
El desarrollo del Alzheimer se asocia con cambios físicos y químicos en el cerebro, incluida la acumulación de placas de proteínas llamadas beta-amiloide y ovillos neurofibrilares compuestos de proteína tau. Estas alteraciones provocan la muerte de células nerviosas y la disrupción de las conexiones entre ellas, lo que resulta en la progresiva pérdida de funciones cognitivas y conductuales.

Aunque no existe cura para el Alzheimer, existen tratamientos farmacológicos y terapias no farmacológicas que pueden ayudar a aliviar los síntomas y mejorar la calidad de vida de los pacientes en las etapas tempranas y moderadas de la enfermedad. Sin embargo, a medida que avanza la enfermedad, el enfoque se centra más en la atención y el apoyo integral, tanto para la persona afectada como para sus cuidadores y familiares.

El Alzheimer es una enfermedad desgarradora que afecta no solo a quienes la padecen, sino también a sus seres queridos. La investigación continua es fundamental para comprender mejor sus mecanismos subyacentes, desarrollar tratamientos más efectivos y, en última instancia, encontrar una cura para esta enfermedad que roba la memoria y la identidad de quienes la sufren.
\subsection{Terapias de reminiscencia}
La reminiscencia, según la definición de la CEAFA (Confederación Española de Asociaciones de Familiares de personas con Alzheimer y otras demencias), es una técnica que busca evocar recuerdos en las personas, especialmente aquellos relacionados con eventos importantes de su vida.

En las terapias de reminiscencia se emplean diversos materiales, como fotografías, vídeos, noticias de periódicos, audios y objetos significativos, con el objetivo de estimular el recuerdo y activar la memoria a través de las emociones que despiertan estos elementos en el paciente.

Además de estimular los cinco sentidos para provocar el recuerdo, también se utiliza la narración de historias conocidas por el paciente. Por lo tanto, es crucial conocer las experiencias pasadas del individuo afectado para adaptar los materiales utilizados en las terapias.

La participación de personas cercanas al paciente, como familiares y cuidadores, en ejercicios de reminiscencia puede mejorar significativamente su calidad de vida.

Es recomendable construir relatos basados en las historias de vida del paciente para estos ejercicios de memoria. Las terapias de reminiscencia se consideran como parte de un proceso más amplio en el que el individuo intenta recuperar y vincular recuerdos que abarcan gran parte de su vida.
\subsection{Historia de vida}
Las Historias de Vida son registros detallados de los aspectos más relevantes de la vida de un paciente, o de personas significativas para él. Son especialmente importantes en las primeras fases del Alzheimer, cuando la memoria aún es relativamente intacta. Estas historias proporcionan dignidad al paciente y permiten a quienes lo rodean conocer mejor su identidad, a medida que las pérdidas de memoria comienzan a ser significativas.

La creación de estas historias es vital para preservar la identidad del paciente. Pueden incluir detalles sobre la familia, la carrera profesional, los viajes y otros aspectos importantes de la vida del individuo. La elaboración puede realizarse de diversas formas, como escribir un libro, crear collages de fotos, producir una película o usar una "caja de memoria".

Es fundamental que estas historias reflejen la perspectiva personal del paciente, incluyendo emociones, sentimientos e interpretaciones. No se trata simplemente de relatar hechos cronológicos, sino de capturar la esencia única de la persona.

Un terapeuta suele encargarse de recopilar y estructurar los eventos importantes de la vida del paciente para crear la Historia de Vida. La estructura puede variar según las necesidades del paciente y los objetivos terapéuticos.

La participación de familiares o conocidos puede facilitar la evocación de recuerdos y enriquecer la Historia de Vida. Además, fortalece la comunicación entre el paciente y sus seres queridos, facilitando el proceso terapéutico.

\section{Objetivos}
\label{sec:objetivos}
Este trabajo  busca desarrollar un chatbot que permita a los terapeutas aplicar terapia de reminiscencia a sus pacientes. En concreto, se centra en la primera etapa de búsqueda de la información con el objetivo de, más adelante generar las historias de vida y aplicar la terapia. 

Para cumplir este objetivo general, se abordan los siguientes objetivos específicos:

 \begin{itemize}
 	
\item Desarrollar un primer chatbot básico capaz de realizar preguntas predefinidas y almacenar las respuestas de forma eficiente.
 
 \item Estudiar las técnicas modernas de procesamiento del lenguaje como los LLMs, bibliotecas como NLTK y spaCy. Además, de las diferentes APIs de para el desarrollo de chatbots. 
 
\item Mejorar el primer chatbot haciendolo más inteligente y capaz de analizar las respuestas, identificar la información omitida y hacer preguntas específicas para obtener la información faltante.
 
\item Hacer una versión final del chatbot que sea capaz de analizar las repuestas y sepa tirar del hilo. También que sea capaz de generar preguntas adecuadas. 

\item Generar una interfaz sencilla de usar para que haga del chatbot una herramienta útil para el próposito para el que ha sido desarrollada.

\item Ordenar la información obtenida para que sea fácil de analizar, entender y procesar, con la intención de generar historias de vida a partir de ella. 

\end{itemize}

Para llevar el control de versiones utilizaremos el repositorio de github:\\
 https://github.com/NILGroup/TFG-2324-ChatbotCANTOR. 

\section{Estructura de la memoria}
El presente documento esta formado por 5  capítulos, incluyendo el presente capítulo. Estos se organizan como sigue. 
\begin{itemize}
	\item \textbf{Capítulo 1: Introducción}. El capítulo presente, se presenta el proyecto, motivación, objetivos y estructura del mismo. 
	\item \textbf{Capítulo 2: Estado de la cuestión}. Es segundo capítulo, nos muestra una aproximación a las historias de vida, la enfermedad del Alzhéimer y la terapia de remiscencia. También muestra otros trabajos relacionados, contexto y precedentes del proyecto. 
	\item \textbf{Capítulo 3: Marco teórico} En este tercer capítulo se explica el estudio profundo de las herramientas de procesamiento del lenguaje, interfaces y almacenamiento de la información que se llevo a cabo para decidir cómo implementar el chatbot. Así se explican las diferentes opciones y los motivos que llevaron a elegir cada una de ellas. 
	\item \textbf{Capítulo 4: Desarrollo del chatbot}
	Este capítulo cuenta cómo se ha ido desarrollando el proyecto, explicando cada una de las versiones que se han llevado a cabo. Se explica, desde la primera versión, los problemas que se han ido encontrando, las funcionalidades extra añadidas etc. 
	\item \textbf{Capítulo 5: Versión final y resultados}
	En el capítulo 5, se presenta la interfaz final con todas sus funcionalidades. El software y las herramientas utilizados, la guía para la instalación y puesta en marcha, la arquitectura del sistema final etc. 
	\item \textbf{Capítulo 6: Conclusiones y trabajo futuro}
	En el capítulo 5, se presenta la interfaz final con todas sus funcionalidades. El software y las herramientas utilizados, la guía para la instalación y puesta en marcha, la arquitectura del sistema final etc. 
	\item \textbf{Anexos} Adicionalmente, en el anexo se muestra una conversación con un paciente real para ver el funcionamiento del chatbot. 
	
\end{itemize}



