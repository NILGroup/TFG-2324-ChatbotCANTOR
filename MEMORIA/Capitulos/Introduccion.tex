\chapter{Introducción}
\label{cap:introduccion}

\chapterquote{Hay enfermos incurables, pero ninguno incuidable}{Francesc Torralba}

El constante aumento en la esperanza de vida a nivel mundial ha dado lugar a una población cada vez más envejecida. Este hecho ha traído consigo un aumento en las enfermedades asociadas con la vejez, como el Alzheimer y otros tipos de demencia. Los esfuerzos de la investigación en estos campos se han centrado en desarrollar tratamientos que puedan ralentizar el avance de esta enfermedad. La memoria juega un papel fundamental en la formación de nuestra identidad a través de nuestras experiencias y vivencias. Su pérdida puede ser devastadora, ya que afecta profundamente a la calidad de vida de la persona y de quienes la rodean.

Entre las estrategias no farmacológicas para el tratamiento del deterioro cognitivo y la memoria, destaca la terapia de reminiscencia. Esta técnica busca estimular los recuerdos pasados del paciente, lo que puede tener beneficios significativos para su bienestar social, mental y emocional.

Por otro lado, la inteligencia artificial está cada día más presente en nuestra vida diaria. Tiene una gran variedad de usos y aplicaciones que abarcan desde asistentes virtuales en nuestros teléfonos inteligentes hasta sistemas de recomendación en plataformas de streaming y comercio electrónico. Además, la inteligencia artificial se utiliza en la medicina para diagnósticos más precisos, en la industria para automatizar procesos de fabricación e incluso en la conducción autónoma de vehículos. Esta omnipresencia de la inteligencia artificial está transformando profundamente diversos sectores, impulsando la eficiencia, la personalización de servicios y generando nuevas oportunidades de desarrollo tecnológico.  

El presente proyecto trata de encontrar la forma de aplicar la inteligencia artificial, y en concreto el procesamiento del lenguaje a la terapia de reminiscencia. En concreto, se busca desarrollar una herramienta conversacional de apoyo para la realización de este tipo de terapia. 

Este capítulo tiene tres objetivos principales. En primer lugar, explicar cuál es la motivación actual con la que parte este problema. En segundo lugar, dejar claros cuáles son los objetivos que se plantean como punto de partida del proyecto. Finalmente, con la intención de dar una visión global del proyecto, que permita entenderlo y hacerse una idea de como es la presente memoria, habrá una sección final en la que se explicará la estructura de este documento, haciendo hincapié en qué se puede encontrar en cada capítulo.  

\section{Motivación}

A pesar de los avances en la investigación médica, todavía no se ha encontrado una cura definitiva para el Alzheimer. La mayoría de los tratamientos se centran en aliviar los síntomas y ralentizar la progresión de la enfermedad. Entre estos tratamientos, la terapia de reminiscencia ha surgido como una opción no farmacológica destacada.

Esta terapia se centra en estimular los recuerdos del pasado del paciente, lo que puede tener beneficios significativos en su bienestar social, mental y emocional. Ayuda a las personas a recordar y compartir experiencias pasadas, lo que puede ser reconfortante y estimulante, especialmente para aquellos que enfrentan el desafío de la pérdida de memoria asociada con el Alzheimer.

Durante el período académico 2023-2024, se desarrolló la aplicación \textit{chatbot yayo} con el objetivo de facilitar a los terapeutas la realización de terapias basadas en reminiscencia, simplificando y agilizando el proceso. Este desarrollo parte desde cero y busca crear un chatbot funcional y útil, capaz de mantener una conversación útil con el paciente para extraer la información necesaria.

Este Trabajo de Fin de Grado tiene como propósito asistir a terapeutas, familiares o amigos de pacientes con demencia en la obtención del material necesario para llevar a cabo terapias de reminiscencia. Se busca mejorar la eficacia de estas terapias y, en consecuencia, la calidad de vida tanto de los pacientes como de sus familiares. 

\subsection{Enfermedad de Alzheimer}
La enfermedad de Alzheimer (EA) es la principal causa de demencia en adultos mayores. La EA es una enfermedad compleja que viene determinada por múltiples factores. A veces es hereditaria y está caracterizada por la pérdida de neuronas y sinapsis, junto con la presencia de placas seniles y degeneración neurofibrilar. Clínicamente, se manifiesta como una demencia progresiva, comenzando con fallas en la memoria reciente y avanzando hasta la dependencia total del paciente.

Según \cite{Donoso2003} La incidencia de la EA aumenta con la edad, siendo rara antes de los 50 años y afectando a un 1-2$\%$ de los sujetos a los 60 años, 3-5$\%$ a los 70, 15-20$\%$ a los 80 y hasta la mitad de los mayores de 85 años. Es más frecuente en mujeres, posiblemente debido a la mayor longevidad.

La enfermedad se desarrolla en distintas etapas. En la etapa inicial, los síntomas pueden pasar desapercibidos o atribuirse a simples descuidos. La persona afectada puede experimentar dificultades para recordar nombres, eventos recientes o encontrar las palabras adecuadas en conversaciones. A pesar de estos desafíos, generalmente conservan la capacidad de realizar tareas cotidianas con cierta independencia. Sin embargo, es posible que comiencen a perder interés en actividades previamente disfrutadas.
	
A medida que la enfermedad progresa, los síntomas se vuelven más evidentes y problemáticos. La pérdida de memoria se vuelve más pronunciada, con dificultades para reconocer a familiares y amigos cercanos. Además, pueden surgir problemas de orientación en tiempo y espacio, lo que puede resultar en desorientación incluso en entornos familiares. Las habilidades de comunicación también se ven afectadas, con dificultades para seguir conversaciones o expresar pensamientos de manera coherente.

Finalmente, el Alzheimer alcanza su punto más devastador en la última etapa. La pérdida de memoria es profunda y completa, con una incapacidad para recordar incluso eventos recientes o reconocer caras familiares. La persona afectada puede experimentar cambios significativos en la personalidad y el comportamiento, volviéndose agitada, ansiosa o incluso agresiva en ocasiones. La capacidad para realizar actividades básicas de la vida diaria, como vestirse o alimentarse, se ve seriamente comprometida, y la supervisión constante se vuelve esencial.

El tratamiento farmacológico de la enfermedad de Alzheimer (EA) se basa en el uso de medicamentos para mejorar los defectos cognitivos y corregir trastornos conductuales. Los fármacos más valorados son los inhibidores de la acetilcolinesterasa, que intentan compensar la pérdida de neuronas colinérgicas en la corteza cerebral. El tratamiento también abarca fármacos para tratar trastornos conductuales, como antidepresivos, tranquilizantes o inductores del sueño (\cite{chung2000neurobehavioral}).

Además del tratamiento farmacológico, la actividad física y mental es fundamental para estimular la actividad cerebral y prevenir la pérdida de memoria. La orientación familiar es esencial para ayudar a los familiares a manejar los trastornos conductuales y aumentar la calidad de vida de los pacientes con EA. Un ejemplo de terapia no farmacológica que ayuda a mantener la actividad mental es la terapia de reminiscencia.


\subsection{Terapias de reminiscencia}
La reminiscencia, según la definición de la CEAFA (Confederación Española de Asociaciones de Familiares de personas con Alzheimer y otras demencias), es una técnica que busca evocar recuerdos en las personas, especialmente aquellos relacionados con eventos importantes de su vida.

En las terapias de reminiscencia se emplean diversos materiales, como fotografías, vídeos, noticias de periódicos, audios y objetos significativos, con el objetivo de estimular el recuerdo y activar la memoria a través de las emociones que despiertan estos elementos en el paciente.

Además de estimular los cinco sentidos para provocar el recuerdo, también se utiliza la narración de historias conocidas por el paciente. Por lo tanto, es crucial conocer las experiencias pasadas del individuo afectado para adaptar los materiales utilizados en las terapias. La participación de personas cercanas al paciente, como familiares y cuidadores, en ejercicios de reminiscencia puede mejorar significativamente su calidad de vida.

Es recomendable construir relatos basados en las historias de vida del paciente para estos ejercicios de memoria. Las terapias de reminiscencia se consideran como parte de un proceso más amplio en el que el individuo intenta recuperar y vincular recuerdos que abarcan gran parte de su vida.

Una revisión sobre los posibles efectos de la terapia de reminiscencia en personas con demencia y sus cuidadores concluyó que disminuía las alteraciones de conducta y los síntomas depresivos, y que mejoraba la cognición (\cite{huang2015reminiscence}). En otra revisión se observaron mejorías en el estado de ánimo, la habilidad cognitiva, la conducta social y el bienestar general (\cite{cotelli2012reminiscence}), y recientemente se encontró que la terapia de reminiscencia mejoraba la salud física y que
aumentaba la participación de los pacientes (\cite{irazoki2027eficacia}).
\subsection{Historia de vida}
Las Historias de Vida son registros detallados de los aspectos más relevantes de la vida de un paciente, o de personas significativas para él. Son especialmente importantes en las primeras fases del Alzheimer, cuando la memoria aún es relativamente intacta. Estas historias proporcionan dignidad al paciente y permiten a quienes lo rodean conocer mejor su identidad, a medida que las pérdidas de memoria comienzan a ser significativas.

La creación de estas historias es vital para preservar la identidad del paciente. Pueden incluir detalles sobre la familia, la carrera profesional, los viajes y otros aspectos importantes de la vida del individuo. La elaboración puede realizarse de diversas formas, como escribir un libro, crear collages de fotos, producir una película o usar una ``caja de memoria''.

Es fundamental que estas historias reflejen la perspectiva personal del paciente, incluyendo emociones, sentimientos e interpretaciones. No se trata simplemente de relatar hechos cronológicos, sino de capturar la esencia única de la persona.

Un terapeuta suele encargarse de recopilar y estructurar los eventos importantes de la vida del paciente para crear la Historia de Vida. La estructura puede variar según las necesidades del paciente y los objetivos terapéuticos.

La participación de familiares o conocidos puede facilitar la evocación de recuerdos y enriquecer la Historia de Vida. Además, fortalece la comunicación entre el paciente y sus seres queridos, facilitando el proceso terapéutico.

\section{Objetivos}
\label{sec:objetivos}
Este trabajo  busca desarrollar un chatbot que permita a los terapeutas aplicar terapia de reminiscencia a sus pacientes. En concreto, se centra en la primera etapa de búsqueda de la información con el objetivo de, más adelante generar las historias de vida y aplicar la terapia. 

Para cumplir este objetivo general, se abordan los siguientes objetivos específicos:

 \begin{itemize}
 	
\item Desarrollar un primer chatbot básico capaz de realizar preguntas predefinidas y almacenar las respuestas de forma eficiente.
 
 \item Estudiar las técnicas modernas de procesamiento del lenguaje como los LLMs, bibliotecas como NLTK y spaCy. Además, de las diferentes APIs de para el desarrollo de chatbots. 
 
\item Mejorar el primer chatbot haciendolo más inteligente y capaz de analizar las respuestas, identificar la información omitida y hacer preguntas específicas para obtener la información faltante.
 
\item Hacer una versión final del chatbot que sea capaz de analizar las repuestas y sepa tirar del hilo. También que sea capaz de generar preguntas adecuadas. 

\item Generar una interfaz sencilla de usar para que haga del chatbot una herramienta útil para el próposito para el que ha sido desarrollada.

\item Ordenar la información obtenida para que sea fácil de analizar, entender y procesar, con la intención de generar historias de vida a partir de ella. 

\end{itemize}

Para llevar el control de versiones utilizaremos el repositorio de github:\\
 https://github.com/NILGroup/TFG-2324-ChatbotCANTOR. 

\section{Estructura de la memoria}
El presente documento esta formado por 5  capítulos, incluyendo el presente capítulo. Estos se organizan como sigue. 
\begin{itemize}
	\item \textbf{Capítulo 1: Introducción}. El capítulo presente, se presenta el proyecto, motivación, objetivos y estructura del mismo. 
	\item \textbf{Capítulo 2: Estado de la cuestión}. Es segundo capítulo, nos muestra una aproximación a las historias de vida, la enfermedad del Alzhéimer y la terapia de remiscencia. También muestra otros trabajos relacionados, contexto y precedentes del proyecto. 
	\item \textbf{Capítulo 3: Tecnologías utilizadas} En este tercer capítulo se explica el estudio profundo de las herramientas de procesamiento del lenguaje, interfaces y almacenamiento de la información que se llevo a cabo para decidir cómo implementar el chatbot. Así se explican las diferentes opciones y los motivos que llevaron a elegir cada una de ellas. 
	\item \textbf{Capítulo 4: Desarrollo de prototipos}
	Este capítulo cuenta cómo se ha ido desarrollando el proyecto, explicando cada una de las versiones que se han llevado a cabo. Se explica, desde la primera versión, los problemas que se han ido encontrando, las funcionalidades extra añadidas etc. 
	\item \textbf{Capítulo 5: Diseño de un chatbot de ayuda a la terapia de reminiscencia}
	En el capítulo 5, se presenta la herramienta final con todas sus funcionalidades. El software y las herramientas utilizados, la guía para la instalación y puesta en marcha, la arquitectura del sistema final etc. 
	\item \textbf{Capítulo 6: Conclusiones y trabajo futuro}
	En el capítulo 5, se presenta la interfaz final con todas sus funcionalidades. El software y las herramientas utilizados, la guía para la instalación y puesta en marcha, la arquitectura del sistema final etc. 
	%\item \textbf{Anexos} Adicionalmente, en el anexo se muestra una conversación con un paciente real para ver el funcionamiento del chatbot. 
	
\end{itemize}



