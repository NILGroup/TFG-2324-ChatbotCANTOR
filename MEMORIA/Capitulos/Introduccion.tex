\chapter{Introducción}
\label{cap:introduccion}

\chapterquote{Hay enfermos incurables, pero ninguno incuidable}{Francesc Torralba}

En este primer capítulo se muestran los motivos que me han llevado a la realización de este trabajo y los objetivos que se buscaban alcanzar desde el punto de partida. 


\section{Motivación}

En la actualidad, la pérdida de memoria afecta a un amplio sector de la población, desde personas con deterioro cognitivo leve hasta aquellos que enfrentan demencias más severas, como el Alzheimer. Esta condición no solo impacta la calidad de vida del paciente, sino que también afecta el bienestar de sus seres queridos. Según datos de la Sociedad Española de Neurología, en España hay 800 mil personas que sufren esta enfermedad.

Se ha observado que las técnicas no farmacológicas ofrecen resultados muy alentadores en la preservación de la memoria, el mantenimiento de las habilidades cognitivas y la retención de recuerdos, lo que ayuda a retrasar el deterioro cognitivo. Entre estas técnicas, los enfoques basados en la revisión de la propia Historia de Vida de la persona afectada han demostrado ser altamente efectivos. Estos implican que el paciente, incluso aquel que padece demencia, registre personalmente las experiencias, personas y lugares más significativos de su vida. El propósito es fomentar la conversación sobre diversos temas, eventos pasados y acontecimientos históricos. Estos ejercicios han demostrado ser beneficiosos para preservar habilidades como el razonamiento, la autoestima, la confianza y las habilidades sociales. Hasta ahora, los terapeutas que implementan estas terapias de reminiscencia han tenido que crear manualmente las historias de vida de los pacientes y depender de documentos impresos para preparar las sesiones.

Durante el período académico 2023-2024, se desarrolló la aplicación YayoBot con el objetivo de facilitar a los terapeutas la realización de terapias basadas en reminiscencia, simplificando y agilizando el proceso. Este desarrollo parte desde cero y busca crear un chatbot funcional y útil, capaz de mantener una conversación útil con el paciente para extraer la información necesaria.

Este Trabajo de Fin de Grado tiene como propósito asistir a terapeutas, familiares o amigos de pacientes con demencia en la obtención del material necesario para llevar a cabo terapias de reminiscencia. Se busca mejorar la eficacia de estas terapias y, en consecuencia, la calidad de vida tanto de los pacientes como de sus familiares. 


\section{Objetivos}

Este trabajo tiene como objetivo conseguir desarrollar desde cero un chatbot usando la API de Bard que  sea capaz de: \begin{enumerate}
\item Desarrollar un primer chatbot básico capaz de contestar a preguntas predefinidas y almacenar respuestas. 
\item Mejorar el anterior chatbot haciendolo más inteligente y capaz de analizar las respuestas, identificar la información omitida , hacer preguntas específicas para obtener la información que falta etc. 
\item Hacer una versión final del chatbot que sea capaz de analizar las repuestas y sepa tirar del hilo. También que sea capaz de generar preguntas adecuadas. 
\end{enumerate}

\section{Plan de trabajo}

El plan de trabajo para obtener alcanzar los objetivos desarrollados anteriormente será: \begin{enumerate}
	\item Desarrollar la primera versión del chatbot antes del 15 de Noviembre, habiendo trabajado para entonces la introducción de la memoria.
	\item Tener para finales de Febrero, la segunda versión desarrollada así como un 85\% de la memoria del proyecto, principalmente los capítulos centrales.  
	\item Desarrollar para Mayo la versión final. Tener el código cerrado y la memoria con todas las conclusiones, resultados y trabajo futuro. 
	
\end{enumerate}

Para llevar el control de versiones utilizaremos el repositorio de github:\\
 https://github.com/NILGroup/TFG-2324-ChatbotCANTOR. 

