\chapter{Introducción}
\label{cap:introduccion}


\chapterquote{Hay enfermos incurables, pero ninguno incuidable}{Francesc Torralba}


El avance en el campo de la medicina en el último siglo ha permitido un notable aumento en la esperanza de vida a nivel mundial. Sin embargo, este alargamiento de la vida también ha traído consigo un aumento en las enfermedades relacionadas con la vejez, como el Alzheimer. A medida que vivimos más tiempo, enfrentamos un mayor riesgo de desarrollar estas condiciones.


\section{Motivación}

A pesar de los avances en la investigación médica, todavía no se ha encontrado una cura definitiva para el Alzheimer. La mayoría de los tratamientos se centran en aliviar los síntomas y ralentizar la progresión de la enfermedad. Entre estos tratamientos, la terapia de reminiscencia ha surgido como una opción no farmacológica destacada.

La terapia de reminiscencia se centra en estimular los recuerdos del pasado del paciente, lo que puede tener beneficios significativos en su bienestar social, mental y emocional. Ayuda a las personas a recordar y compartir experiencias pasadas, lo que puede ser reconfortante y estimulante, especialmente para aquellos que enfrentan el desafío de la pérdida de memoria asociada con el Alzheimer.

Durante el período académico 2023-2024, se desarrolló la aplicación YayoBot con el objetivo de facilitar a los terapeutas la realización de terapias basadas en reminiscencia, simplificando y agilizando el proceso. Este desarrollo parte desde cero y busca crear un chatbot funcional y útil, capaz de mantener una conversación útil con el paciente para extraer la información necesaria.

Este Trabajo de Fin de Grado tiene como propósito asistir a terapeutas, familiares o amigos de pacientes con demencia en la obtención del material necesario para llevar a cabo terapias de reminiscencia. Se busca mejorar la eficacia de estas terapias y, en consecuencia, la calidad de vida tanto de los pacientes como de sus familiares. 


\section{Objetivos}
Este trabajo  busca desarrollar un chatbot que permita a los terapeutas aplicar terapia de reminiscencia a sus pacientes. En concreto, se centra en la primera etapa de búsqueda de la información con el objetivo de, más adelante generar las historias de vida y aplicar la terapia. 

Para cumplir este objetivo general, se abordan los siguientes objetivos específicos:

 \begin{itemize}
 	
\item Desarrollar un primer chatbot básico capaz de realizar preguntas predefinidas y almacenar las respuestas de forma eficiente.
 
 \item Estudiar las técnicas modernas de procesamiento del lenguaje como los LLMs, bibliotecas como NLTK y spaCy. Además, de las diferentes APIs de para el desarrollo de chatbots. 
 
\item Mejorar el primer chatbot haciendolo más inteligente y capaz de analizar las respuestas, identificar la información omitida y hacer preguntas específicas para obtener la información faltante.
 
\item Hacer una versión final del chatbot que sea capaz de analizar las repuestas y sepa tirar del hilo. También que sea capaz de generar preguntas adecuadas. 

\end{itemize}

\section{Plan de trabajo}

El plan de trabajo para obtener alcanzar los objetivos desarrollados anteriormente será: \begin{enumerate}
	\item Desarrollar la primera versión del chatbot antes del 15 de Noviembre, habiendo trabajado para entonces la introducción de la memoria.
	\item Tener para finales de Febrero, la segunda versión desarrollada así como un 85\% de la memoria del proyecto, principalmente los capítulos centrales.  
	\item Desarrollar para Mayo la versión final. Tener el código cerrado y la memoria con todas las conclusiones, resultados y trabajo futuro. 
	
\end{enumerate}

Para llevar el control de versiones utilizaremos el repositorio de github:\\
 https://github.com/NILGroup/TFG-2324-ChatbotCANTOR. 

\section{Estructura de la memoria}