\chapter*{Introduction}
\addcontentsline{toc}{chapter}{Introduction}
\label{chap:introduction}

\chapterquote{There are incurable patients, but none uncareable}{Francesc Torralba}

The constant increase in life expectancy worldwide has led to an increasingly aging population. This has brought with it an increase in diseases associated with old age, such as Alzheimer's and other types of dementia. Research efforts in these fields have focused on developing treatments that can slow the progression of this disease. Memory plays a fundamental role in shaping our identity through our experiences. Its loss can be devastating as it profoundly affects the quality of life of the person and those around them.

Among non-pharmacological strategies for treating cognitive decline and memory loss, reminiscence therapy stands out. This technique aims to stimulate the patient's past memories, which can have significant benefits for their social, mental, and emotional well-being.

On the other hand, artificial intelligence is increasingly present in our daily lives. It has a wide variety of uses and applications ranging from virtual assistants on our smartphones to recommendation systems on streaming platforms and e-commerce. Additionally, artificial intelligence is used in medicine for more accurate diagnoses, in industry to automate manufacturing processes, and even in autonomous vehicle driving. This omnipresence of artificial intelligence is profoundly transforming various sectors, driving efficiency, service personalization, and generating new technological development opportunities.

This project aims to find a way to apply artificial intelligence, specifically natural language processing, to reminiscence therapy. Specifically, it seeks to develop a conversational support tool for conducting this type of therapy.

This chapter has three main objectives. First, to explain the current motivation behind this problem. Second, to clarify the goals set as the project's starting point. Finally, with the intention of providing an overview of the project, there will be a final section explaining the structure of this document, emphasizing what can be found in each chapter.

\section{Motivation}

Despite advances in medical research, no definitive cure for Alzheimer's has yet been found. Most treatments focus on alleviating symptoms and slowing the progression of the disease. Among these treatments, reminiscence therapy has emerged as a prominent non-pharmacological option.

This therapy focuses on stimulating the patient's past memories, which can have significant benefits for their social, mental, and emotional well-being. It helps people recall and share past experiences, which can be comforting and stimulating, especially for those facing the challenge of memory loss associated with Alzheimer's.

During the 2023-2024 academic year, a chatbot was developed with the aim of facilitating therapists in conducting reminiscence-based therapies, simplifying and streamlining the process. This development starts from scratch and seeks to create a functional and useful chatbot capable of maintaining a useful conversation with the patient to extract the necessary information.

This Final Degree Project aims to assist therapists, family members, or friends of patients with dementia in obtaining the necessary material for conducting reminiscence therapies. It seeks to improve the effectiveness of these therapies and, consequently, the quality of life for both patients and their families.

\subsection{Alzheimer's Disease}
Alzheimer's disease (AD) is the leading cause of dementia in older adults. AD is a complex disease determined by multiple factors. Sometimes it is hereditary. It is characterized by the loss of neurons and synapses, along with the presence of amyloid plaques and neurofibrillary degeneration. Clinically, it manifests as progressive dementia, starting with recent memory failures and advancing to total dependency.

According to \cite{Donoso2003}, the incidence of AD increases with age, being rare before the age of 50 and affecting 1-2$\%$ of individuals at age 60, 3-5$\%$ at age 70, 15-20$\%$ at age 80, and up to half of those over 85. It is more frequent in women, possibly due to their greater longevity.

The disease develops in different stages. In the initial stage, symptoms may go unnoticed or be attributed to simple forgetfulness. The affected person may experience difficulties remembering names, recent events, or finding the right words in conversations. Despite these challenges, they generally retain the ability to perform daily tasks with some independence. However, they may begin to lose interest in previously enjoyed activities.

As the disease progresses, the symptoms become more evident and problematic. Memory loss becomes more pronounced, with difficulties recognizing close family and friends. Additionally, problems with time and space orientation may arise, leading to disorientation even in familiar environments. Communication skills are also affected, with difficulties following conversations or expressing thoughts coherently.

Finally, Alzheimer's reaches its most devastating stage. Memory loss is profound and complete, with an inability to recall even recent events or recognize familiar faces. The affected person may experience significant changes in personality and behavior, becoming agitated, anxious, or even aggressive at times. The ability to perform basic daily life activities, such as dressing or feeding, is severely compromised, and constant supervision becomes essential.

The pharmacological treatment of Alzheimer's disease (AD) is based on the use of medications to improve cognitive defects and correct behavioral disorders. The most valued drugs are acetylcholinesterase inhibitors, which attempt to compensate for the loss of cholinergic neurons in the cerebral cortex. Treatment also includes drugs to treat behavioral disorders, such as antidepressants, tranquilizers, or sleep inducers (\cite{chung2000neurobehavioral}).

In addition to pharmacological treatment, physical and mental activity is crucial for stimulating brain activity and preventing memory loss. Family guidance is essential to help family members manage behavioral disorders and increase the quality of life for AD patients. An example of a non-pharmacological therapy that helps maintain mental activity is reminiscence therapy.

\subsection{Reminiscence Therapy}
Reminiscence, according to the definition by CEAFA (Spanish Confederation of Associations of Relatives of People with Alzheimer's and Other Dementias), is a technique that seeks to evoke memories in people, especially those related to significant events in their lives.

Reminiscence therapies use various materials such as photographs, videos, newspaper clippings, audios, and significant objects to stimulate memory and activate it through the emotions these elements evoke in the patient.

In addition to stimulating the five senses to trigger memory, storytelling known to the patient is also used. Therefore, it is crucial to know the individual's past experiences to tailor the materials used in the therapies. The participation of people close to the patient, such as family members and caregivers, in reminiscence exercises can significantly improve their quality of life.

It is recommended to build stories based on the patient's life experiences for these memory exercises. Reminiscence therapies are considered part of a broader process in which the individual attempts to recover and link memories spanning much of their life.

A review on the potential effects of reminiscence therapy on people with dementia and their caregivers concluded that it decreased behavioral disturbances and depressive symptoms, and improved cognition (\cite{huang2015reminiscence}). Another review observed improvements in mood, cognitive ability, social behavior, and general well-being (\cite{cotelli2012reminiscence}), and recently, it was found that reminiscence therapy improved physical health and increased patient participation (\cite{irazoki2027eficacia}).

\subsection{Life Stories}
Life Stories are detailed records of the most relevant aspects of a patient's life or significant people in their life. They are especially important in the early stages of Alzheimer's when memory is still relatively intact. These stories provide dignity to the patient and allow those around them to better understand their identity as memory losses become significant.

Creating these stories is vital for preserving the patient's identity. They can include details about family, professional career, hobbies, travels, and other important aspects of the individual's life. They can be created in various forms, such as writing a book, creating photo collages, producing a movie, or using a "memory box."

It is essential that these stories reflect the patient's personal perspective, including emotions, feelings, and interpretations. It is not simply about recounting chronological facts but capturing the unique essence of the person.

A therapist usually undertakes the task of collecting and structuring the significant events of the patient's life to create the Life Story. The structure can vary depending on the patient's needs and therapeutic goals.

The involvement of family members or acquaintances can facilitate memory recall and enrich the Life Story. Additionally, it strengthens communication between the patient and their loved ones, facilitating the therapeutic process.

\section{Objectives and Work Methodology}
\label{sec:objectives}

The objective of this work is to develop a chatbot that supports the conduction of reminiscence therapy. Specifically, the chatbot should meet the following requirements:

\begin{itemize}
	\item Be capable of maintaining a conversation, guiding the flow through predefined topics about which data from the patient is sought.
	\item Extract the desired information from the patient's responses, identify additional and missing information.
	\item Generate questions related to the responses, to obtain any initially missing information.
	\item Have a user-friendly interface, making this chatbot a useful tool.
\end{itemize}


To achieve these objectives, the following methodology will be followed:

\begin{itemize}
	\item Study Alzheimer's disease and reminiscence therapy to understand the characteristics a conversational bot must have to be helpful in this area.
	\item Conduct an in-depth study of various natural language processing tools, from libraries like NLTK and spaCy to different APIs and LLM models. This will allow us to choose the most appropriate tools for subsequent development.
	\item Once the development phase begins, the first step will be to implement a basic chatbot capable of asking predefined questions and efficiently storing responses.
	\item Improve the tool, making it smarter and capable of analyzing responses, identifying omitted information, and asking specific questions to obtain the missing information.
	\item Develop a final version capable of analyzing responses and probing further, in addition to the functionalities of previous prototypes.
	\item Manage the storage of extracted information.
	\item Create a simple and usable interface, making the chatbot a useful tool for its intended purpose.
	\item Allow the user to accompany their responses with images, both to help them give more complete answers and to obtain additional information through processing them.
\end{itemize}

To manage version control, we will use the GitHub repository: \\
https://github.com/NILGroup/TFG-2324-ChatbotCANTOR.

\section{Structure of the Document}
This document consists of 6 chapters, including the current one. They are organized as follows:
\begin{itemize}
	\item \textbf{Chapter 1: Introduction}. The current chapter presents the project, motivation, objectives, and structure.
	\item \textbf{Chapter 2: State of the Art}. This chapter provides an overview of life stories, Alzheimer's disease, and reminiscence therapy. It also presents related works, context, and precedents of the project.
	\item \textbf{Chapter 3: Technologies Used}. This third chapter explains the in-depth study of natural language processing tools, interfaces, and information storage conducted to decide how to implement the chatbot. It explains the different options and the reasons for choosing each.
	\item \textbf{Chapter 4: Prototype Development}. This chapter explains how the project has been developed, describing each version carried out. It details the initial version, the problems encountered, additional functionalities added, etc.
	\item \textbf{Chapter 5: Designing a Reminiscence Therapy Support Chatbot}. In Chapter 5, the final tool with all its functionalities is presented. The software and tools used, the guide for installation and setup, the architecture of the final system, etc.
	\item \textbf{Chapter 6: Conclusions and Future Work}. Chapter 6 discusses the conclusions reached during the project development and the work that could be done in the future to expand it.
\end{itemize}