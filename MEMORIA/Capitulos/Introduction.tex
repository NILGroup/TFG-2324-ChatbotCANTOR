\chapter*{Introduction}
\label{chap:introduction}

\chapterquote{There are incurable patients, but none that are careless}{Francesc Torralba}

The advancement in the field of medicine in the last century has allowed for a notable increase in life expectancy worldwide. However, this lengthening of life has also brought about an increase in age-related diseases, such as Alzheimer's. As we live longer, we face a greater risk of developing these conditions.

\section{Motivation}

Despite advances in medical research, a definitive cure for Alzheimer's has not yet been found. Most treatments focus on alleviating symptoms and slowing the progression of the disease. Among these treatments, reminiscence therapy has emerged as a prominent non-pharmacological option.

Reminiscence therapy focuses on stimulating the patient's memories of the past, which can have significant benefits for their social, mental, and emotional well-being. It helps individuals remember and share past experiences, which can be comforting and stimulating, especially for those facing the challenge of memory loss associated with Alzheimer's.

During the academic period 2023-2024, the YayoBot application was developed with the aim of facilitating therapists in conducting reminiscence-based therapies, simplifying and streamlining the process. This development starts from scratch and aims to create a functional and useful chatbot, capable of maintaining a useful conversation with the patient to extract the necessary information.

This Bachelor's Thesis aims to assist therapists, family members, or friends of dementia patients in obtaining the necessary material to carry out reminiscence therapies. The goal is to improve the effectiveness of these therapies and, consequently, the quality of life of both patients and their families.

\section{Objectives}

This work aims to develop a chatbot that allows therapists to apply reminiscence therapy to their patients. Specifically, it focuses on the first stage of searching for information with the aim of later generating life stories and applying therapy.

To achieve this general objective, the following specific objectives are addressed:

\begin{itemize}
	\item Develop an initial basic chatbot capable of asking predefined questions and storing responses efficiently.
	\item Study modern natural language processing techniques such as LLMs, libraries like NLTK and spaCy, and different APIs for chatbot development.
	\item Improve the initial chatbot, making it smarter and capable of analyzing responses, identifying omitted information, and asking specific questions to obtain the missing information.
	\item Create a final version of the chatbot that can analyze responses and follow up on the conversation. Also, it should be able to generate appropriate questions.
	\item Generate a user-friendly interface to make the chatbot a useful tool for its intended purpose.
	\item Organize the obtained information to make it easy to analyze, understand, and process, with the intention of generating life stories from it.
\end{itemize}

For version control, we will use the GitHub repository: https://github.com/NILGroup/TFG-2324-ChatbotCANTOR.

\section{Structure of the Document}

This document consists of 5 chapters, including this one. They are organized as follows:

\begin{itemize}
	\item \textbf{Chapter 1: Introduction}. The current chapter introduces the project, motivation, objectives, and structure of the thesis.
	\item \textbf{Chapter 2: State of the Art}. This chapter provides an overview of life stories, Alzheimer's disease, reminiscence therapy, as well as other related works, context, and precedents of the project.
	\item \textbf{Chapter 3: Theoretical Framework}. In this third chapter, a deep study of natural language processing tools, interfaces, and information storage was conducted to decide how to implement the chatbot. The different options and the reasons for choosing each one are explained.
	\item \textbf{Chapter 4: Chatbot Development}. This chapter describes how the project has been developed, explaining each of the versions that have been carried out. It explains, from the first version, the problems encountered, the additional functionalities added, etc.
	\item \textbf{Chapter 5: Final Version and Results}. In Chapter 5, the final interface with all its functionalities is presented. The software and tools used, the installation and startup guide, the architecture of the final system, etc., are also discussed.
	\item \textbf{Appendices}. Additionally, in the appendix, a conversation with a real patient is shown to see how the chatbot works.
\end{itemize}













