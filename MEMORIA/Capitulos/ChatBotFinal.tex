\chapter{Herramienta conversacional de ayuda a la terapia a la reminiscencia}
\label{cap:ChatBot final}
Este capítulo tiene como objetivo presentar la solución final desarrollada a partir de la problemática presentada en el capítulo \ref{sec:objetivos}. Para ello, se describirá en profundidad cada uno de los componentes básicos de la arquitectura del sistema, y se presentara esta en sí misma. Por otro lado, se explicará tanto el proceso de puesta en marcha como las herramientas necesarias para ese mismo objetivo. 

Con el objetivo de permitir la opción de estudiar los componentes de forma práctica, es decir, usando el prototipo, este capítulo comenzará por la puesta en marcha. A continuación, se dará una idea global de la arquitectura del sistema y finalmente se irá explorando en cada sección, cada uno de los módulos que la componen. 
\section{Herramientas y puesta en marcha}
Para la puesta en marcha el primer paso es obtener la \textit{API Key} para lo que es necesario el uso de una VPN.
\subsection{VPN}
Debido a las restricciones geográficas actuales de la API de Gemini para poder usarla es necesario el uso de una VPN. En concreto y para el desarrollo de este proyecto la conexión a la VPN se ha realizado mediante la herramienta \textit{hide.me VPN}. Esta herramienta crea un túnel seguro utilizando poderosos protocolos VPN, oculta nuestra IP real con una suya y cifra todo el tráfico de internet que pasa por este túnel para que podamos navegar libremente. Además, \textit{hide.me} está certificada como una VPN cero registros. Esto significa que no se almacena información de ningún tipo. El uso de la VPN es necesario tanto para obtener la \textit{API Key} como para el uso de la misma. Los países en los que se encuentra disponible gemini se pueden consultar en la web \href{https://ai.google.dev/gemini-api/docs/available-regions?hl=es-419} de la api de gemini. 

\subsection{Instalación de la API de Gemini}

Para comenzar a utilizar la API de Gemini con Python, es necesario seguir estos pasos para instalar el SDK y configurar tu clave de API.

En primer lugar, instalamos el SDK (Software Development Kit). La API de Gemini está contenida en el paquete \texttt{google-generativeai} en PyPI, por lo que el primer paso sera instalar esa dependencia.

\begin{lstlisting}[language=Python]
	!pip install -U google-generativeai
\end{lstlisting}

Para utilizar la API de Gemini, se necesita una clave de API obtenida del \href{https://aistudio.google.com/app/apikey} Google AI Studio. Una vez que tengas tu clave, puedes configurarla para que el SDK la utilice:

\begin{lstlisting}[language=Python]
	import google.generativeai as genai
	from google.colab import userdata
	
	GOOGLE_API_KEY = userdata.get('GOOGLE_API_KEY')
	genai.configure(api_key=GOOGLE_API_KEY)
\end{lstlisting}

\section{Arquitectura}

\subsection{Interfaz e interacción con el usuario}


%\begin{itemize}
%	\item Se importan los módulos necesarios, incluyendo el módulo `telebot` para interactuar con la API de Telegram, así como los módulos personalizados `tfg` e `imagenes` para la lógica del chatbot.
	
%	\item Se crea una instancia del bot de Telegram utilizando la API token proporcionada en el archivo `config.py`.
	
%	\item Se definen tres manejadores de mensajes:
%	\begin{itemize}
%		\item El manejador `cmd_start()` se activa cuando un usuario envía el comando `/start` al bot. Este manejador carga una serie de preguntas utilizando la función `cargar_preguntas()` del módulo `tfg` y envía un mensaje de bienvenida al usuario.
		
%		\item El manejador `bot_mensajes_text()` se activa cuando un usuario envía un mensaje de texto al bot. Si el mensaje comienza con %"/", se envía un mensaje de error indicando que el comando no está disponible. Si no, se llama a la función `siguientePregunta()` del módulo `tfg` para procesar la respuesta del usuario y generar una respuesta apropiada.
		
%		\item El manejador `photo()` se activa cuando un usuario envía una foto al bot. El bot descarga la foto, la guarda en el sistema de archivos y llama a la función `analizador_imagenes()` del módulo `imagenes` para analizar la imagen y generar una respuesta apropiada.
%	\end{itemize}
	
%	\item Finalmente, se inicia el bucle de escucha del bot (`bot.infinity_polling()`) para que esté constantemente esperando y respondiendo a los mensajes de los usuarios.
%\end{itemize}

\subsection{Almacenamiento y manejo de la información}

%IDEA AÑADIR SECCIÓN EVALUACIÓN
\subsection{Generación de historias de vida}