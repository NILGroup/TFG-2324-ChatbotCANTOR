\chapter*{Conclusions and Future Work}
\label{chap:conclusions}
This chapter describes the conclusions reached after months of work on a project to build a conversational tool to support reminiscence therapy.

Despite the exponential development of AI in the field of language processing, the provisionality of models and constant changes in legislation across different regions have caused various issues, leading to a more extensive study of the current situation and different alternatives.

In this chapter, we aim not only to present the conclusions about the developed tool but also to share all conclusions associated with the prior theoretical study.

Furthermore, the potential future work that can be considered as a result of the research will be described.

\section{Final Result}
As a result of this project, a conversational tool has been created to support reminiscence therapy.

The tool consists of a chatbot capable of maintaining a conversation with the user, aiming to gather as much information as possible.

The tool can identify which information has not been obtained and generate coherent questions within the context of the conversation to gather the missing data. Additionally, it provides feedback that makes the chat a more meaningful conversation.

Images can be sent to the chatbot at any time, allowing it to provide a description and an associated question. This helps the user recall things more easily by answering more specific questions about an image they have in front of them.

The user interface is intuitive and easy to use. The fact that it is implemented through Telegram significantly reduces the learning curve and makes it much more usable for all types of users. Since it does not require any special technological knowledge, this tool is accessible not only to therapists but also to family members and patients who are familiar with mobile devices and computers.

Although initially designed as a therapeutic tool to gather information about the patient's life, this tool has multiple uses. For example, portable versions, such as a mobile or tablet application, allow it to be used as entertainment during long journeys, which can be enjoyable for users. Its ease of use also allows patients to use it during moments of loneliness, having interesting conversations about their life that stimulate their mind and provide virtual companionship.

The main function of this bot is to seek and store relevant information to generate life stories using other tools. Additionally, at the end of the conversation, a life story constructed with all the obtained information is generated. This result serves as feedback and a reward that keeps the user motivated to complete the conversation.

\section{Future Work}
As a final point of this work, the possible open lines of development and research that have arisen after analyzing the project and that could be studied to improve it are presented.

Firstly, although this work generates a small life story as the final result of the conversation, it would be interesting to connect this work with the project explained in section \ref{sec:trabajocristina} "Life Story Generation Using Deep Learning Techniques." For this, it would be beneficial to transform all the information obtained through the conversation with this chatbot into an input set in the appropriate format for the application developed in that work. This way, life stories more faithful to the person's real life would be generated, completing the full cycle of work prior to reminiscence therapy.

Additionally, another modification that could be interesting is adapting the model to multiple languages to expand the application's target audience. The Gemini API is available in different languages, but all predefined questions would need to be translated for user interaction.

The development of the Gemini API and access to new versions will bring new possibilities. Among them, the integration of videos to be sent to the chatbot for analysis to obtain more biographical information stands out. It might also be interesting to consider the alternative of voice recognition so that, in addition to the Telegram interface, a voice interface could be developed where the patient could speak and listen as if it were a phone call.
