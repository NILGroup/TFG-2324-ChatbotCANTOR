\chapter{Conclusiones y Trabajo Futuro}
\label{cap:conclusiones}
En el presente capítulo se describen las conclusiones a las que se han llegado tras meses de trabajo en un proyecto de construcción de una herramienta conversacional de ayuda a la terapia de reminiscencia. Pese al desarrollo exponencial que esta sufriendo la IA en el campo del procesamiento del lenguaje, la provisionalidad de los módelos y los constantes cambios en las legislaciones de los territorios han llevado a diversos problemas y en consecuencia a un estudio mucho más amplio de la situación actual y de las diferentes alternativas. 

En este capítulo, no solo se pretende hablar de las conclusiones acerca de la herramienta desarrollada, si no a todas las conclusiones asociadas al estudio teórico previo. 

Por otro lado, se decribiran los posibles trabajos futuros que pueden considerarse como fruto de la investigación.

\section{Estado actual del procesamiento del lenguaje}
En los últimos años, el procesamiento del lenguaje avanza a pasos agigantados. Esto tiene numerosas ventajas, aunque también implica algunos riesgos a la hora de desarrollar software.  Estos riesgos tienen como consecuencia el desarrollo de nuevas leyes en los diferentes países que regulen el uso de las IAs. 

Por un lado, los modelos e inteligencias artificiales se desarrollan a un ritmo tal, que aparece el término de ``última versión estable''. Cuando se habla de "última versión estable" en el contexto de los modelos de procesamiento del lenguaje, especialmente en el ámbito del desarrollo de software o modelos de inteligencia artificial (IA) como los modelos de lenguaje, se refiere a la versión más reciente del modelo que ha sido probada y validada para su uso generalizado con una buena calidad y fiabilidad.

Por otro lado, la legislación en los países europeos la frena de alguna manera. Todo esto hace que sea muy inestable. Por ejemplo, mientras el primer cuatri se usaba BARd, luego gemini en otros paises y el 22 de mayo cambiara. 

\section{Herramienta conversacional de ayuda a la terapia de reminiscencia}

\section{Trabajo Futuro}

Aunque la aplicación esta pensada para ser usada por terapeutas junto con los pacientes, la simpleza de la interfaz hace que este en la mano de cualquiera y que pueda ser una herramienta que usen familiares o incluso pacientes. 