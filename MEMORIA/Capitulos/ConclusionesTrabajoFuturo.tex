\chapter{Conclusiones y Trabajo Futuro}
\label{cap:conclusiones}
En el presente capítulo se describen las conclusiones a las que se han llegado tras meses de trabajo en un proyecto de construcción de una herramienta conversacional de ayuda a la terapia de reminiscencia. Pese al desarrollo exponencial que esta sufriendo la IA en el campo del procesamiento del lenguaje, la provisionalidad de los módelos y los constantes cambios en las legislaciones de los territorios han llevado a diversos problemas y en consecuencia a un estudio mucho más amplio de la situación actual y de las diferentes alternativas. 

En este capítulo, no solo se pretende hablar de las conclusiones acerca de la herramienta desarrollada, si no a todas las conclusiones asociadas al estudio teórico previo. 

Por otro lado, se decribiran los posibles trabajos futuros que pueden considerarse como fruto de la investigación.


\section{Resultado final}
Como resultado del desarrollo de este proyecto, se ha creado una herramienta conversacional para apoyar la terapia de reminiscencia. La principal función de este bot es buscar y almacenar información relevante para generar historias de vida utilizando otras herramientas. La interfaz de usuario intuitiva y fácil de usar hace que esta herramienta sea accesible no solo para terapeutas, sino también para familiares y pacientes que estén familiarizados con dispositivos móviles y computadoras.

Aunque inicialmente diseñada como una herramienta terapéutica para recopilar información sobre la vida del paciente, esta herramienta tiene múltiples usos. Por ejemplo, las versiones portátiles, como la aplicación móvil o para tabletas, permiten su uso como entretenimiento durante viajes largos que pueden resultar desafiantes para los pacientes. Además, la facilidad de uso también permite que los pacientes la utilicen en momentos de soledad, manteniendo conversaciones interesantes sobre su vida que estimulan la mente y les proporcionan compañía virtual.

\section{Trabajo futuro}

Como punto final de este trabajo se exponen aquellas posibles líneas abiertas de desarrollo e investigación, que han surgido tras el análisis del proyecto, y que podrían ser estudiados con la finalidad de mejorar el proyecto. 

En primer lugar, aunque este trabajo genere una pequeña historia de vida como resultado final de la conversación sería interesante conectar este trabajo con el proyecto explicado en la sección \ref{sec:trabajocristina} ``Generación de historias de vida usando técnias de Deep Learning''. Para ello sería interesante transformar toda la información que se obtiene mediante la conversación con este chatbot en un conjunto de entrada con el formato adecuado para la aplicación desarrollada en ese trabajo. De esta forma, se generarían historias de vida más fieles a la vida real de la persona, y se completaría el ciclo completo de trabajo previo a la terapia de reminiscencia. 

Por otro lado, otra modificación que podría resultar interesante sería la adaptación del modelo a múltiples lenguajes de forma que se amplié el público objetivo de la aplicación. La API de Gemini esta disponible en diferentes lenguajes, pero el manejo de todas las preguntas predefinidas tendría que ser traducido para la interacción con el usuario. 

El desarrollo de la API de Gemini, y el acceso de nuevas versiones traerá consigo nuevas posibilidades. Entre ellas, destaca la integración de vídeos para enviar al chatbot y que se analicen con el objetivo de obtener más información bibliográfica. También puede ser interesante considerar la alternativa del reconocimiento de voz de forma que además de la interfaz de Telegram, se pudiera desarrollar una interfaz de voz donde el paciente pudiera hablar y escuchar como si se tratará de una llamada telefónica.
