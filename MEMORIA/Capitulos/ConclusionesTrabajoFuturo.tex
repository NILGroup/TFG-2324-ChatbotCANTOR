\chapter{Conclusiones y Trabajo Futuro}
\label{cap:conclusiones}
En el presente capítulo se describen las conclusiones a las que se ha llegado tras meses de trabajo en un proyecto de construcción de una herramienta conversacional de ayuda a la terapia de reminiscencia.

Pese al desarrollo exponencial que esta sufriendo la IA en el campo del procesamiento del lenguaje, la provisionalidad de los módelos y los constantes cambios en las legislaciones de los territorios han ocasionado diversos problemas y en consecuencia a un estudio más extenso de la situación actual y de las diferentes alternativas. 

En este capítulo, no solo se pretende exponer las conclusiones acerca de la herramienta desarrollada, si no todas las conclusiones asociadas al estudio teórico previo. 

Por otro lado, se describirán los posibles trabajos futuros que pueden considerarse fruto de la investigación.


\section{Resultado final}
Como resultado del desarrollo de este proyecto, se ha creado una herramienta conversacional para apoyar la terapia de reminiscencia. 

La herramienta consiste en un chatbot capaz de mantener una conversación con el usuario, con el propósito de obtener tanta información del mismo como sea posible. 

La  herramienta es capaz de identificar qué información no se ha obtenido y generar preguntas coherentes en el contexto de la conversacióny así obtener los datos que faltan. Además, genera \textit{feedback} que hace del chat una conversación con más sentido. 

En cualquier momento se pueden enviar imagenes al chatbot de forma que se obtiene una descripción de las mismas y una pregunta asociada. Esto permite que el usuario pueda recordar cosas con mayor facilidad al hacerse preguntas más específicas sobre una imagen que tiene delante. 

La interfaz de usuario resulta intuitiva y fácil de usar. El hecho de que este implementada a través de Telegram reduce significativamente la curva de aprendizaje y la hace mucho más usable para todo tipo de usuarios. Al no requerir ningún conocimiento tecnológico especial esta herramienta es accesible no solo para terapeutas, sino también para familiares y pacientes que estén familiarizados con dispositivos móviles y computadoras.

Aunque inicialmente diseñada como una herramienta terapéutica para recopilar información sobre la vida del paciente, esta herramienta tiene múltiples usos. Por ejemplo, las versiones portátiles, como la aplicación móvil o para tabletas, permiten su uso como entretenimiento durante viajes largos que pueden resultar amenos para los usuarios. La facilidad de uso también permite que los pacientes la empleen en momentos de soledad, manteniendo conversaciones interesantes sobre su vida que estimulan la mente y les proporcionan compañía virtual.

La principal función de este bot es buscar y almacenar información relevante para generar historias de vida utilizando otras herramientas. Además, al final de la conversación se genera una historia de vida construida con toda la información que ha sido obtenida. Este resultado, supone una forma de \textit{feedback} y una recompensa que mantiene la motivación del usuario para llegar hasta el final de la conversación. 

\section{Trabajo futuro}

Como punto final de este trabajo se exponen aquellas posibles líneas abiertas de desarrollo e investigación, que han surgido tras el análisis del proyecto, y que podrían ser estudiados con la finalidad de mejorar el proyecto. 

En primer lugar, aunque este trabajo genere una pequeña historia de vida como resultado final de la conversación sería interesante conectar este trabajo con el proyecto explicado en la sección \ref{sec:trabajocristina} ``Generación de historias de vida usando técnias de Deep Learning''. Para ello habría que transformar la información que obtiene el chatbot en un conjunto de entrada con el formato adecuado para su procesamiento. De esta forma, se generarían historias de vida más fieles a la vida real de la persona, y se tendrían herramientas para realizar todo el trabajo necesario asociado a la terapia de reminiscencia. 

Por otro lado, otra modificación que podría resultar interesante sería la adaptación del modelo a múltiples lenguajes de forma que se amplié el público objetivo de la aplicación. La API de Gemini esta disponible en diferentes lenguajes con lo que podría mantenerse el uso de esta herramienta. Sin embargo, pero el manejo de todas las preguntas predefinidas tendría que ser traducido para la interacción con el usuario. 

Puede ser interesante considerar la alternativa del reconocimiento de voz de forma que además de la interfaz de Telegram, se pudiera desarrollar una interfaz de voz donde el paciente pudiera hablar y escuchar como si se tratará de una llamada telefónica. Para ello se podrían adaptar algunas funcionalidades de modelos de Gemini. o bibliotecas como ``Whisper'' de OpenAI.

El desarrollo de la API de Gemini, y el acceso de nuevas versiones traerá consigo nuevas posibilidades. Entre ellas, destaca la integración de vídeos para enviar al chatbot y que se analicen con el objetivo de obtener más información bibliográfica.
