\chapter{ChatBot final}
\label{cap:ChatBot final}
\section{Arquitectura}
\section{Herramientas y puesta en marcha}
\subsection{VPN}
\section{Implementación de las funcionalidades}
\subsection{Extracción de la información}
\subsection{Generación de preguntas}
\subsection{Interfaz e interacción con el usuario}
La interfaz de Telegram es intuitiva y fácil de usar, lo que proporciona una experiencia de usuario más agradable y cómoda. Los usuarios pueden interactuar con el chatbot de manera similar a como lo harían con otros contactos o grupos en la aplicación, lo que reduce la curva de aprendizaje y aumenta la aceptación del usuario.

Telegram también ofrece una variedad de funcionalidades integradas que pueden ser útiles para un chatbot de terapia de reminiscencia, como el envío de fotos, videos, archivos y ubicaciones. Esto permite una experiencia de usuario más rica y variada, lo que puede mejorar la efectividad del chatbot en la prestación de servicios de apoyo emocional.

Finalmente, el uso de la plataforma de Telegram simplifica el desarrollo y el mantenimiento del chatbot. No es necesario crear una interfaz de usuario desde cero, ya que Telegram proporciona una interfaz lista para usar. Además, Telegram se encarga de la infraestructura y la gestión del servidor, lo que reduce la carga de trabajo para los desarrolladores y permite centrarse en el desarrollo de funcionalidades específicas para la terapia de reminiscencia.

El siguiente código implementa un chatbot de ayuda a la terapia de reminiscencia utilizando la plataforma de mensajería Telegram. A continuación, se explican las principales funcionalidades del código:

%\begin{itemize}
%	\item Se importan los módulos necesarios, incluyendo el módulo `telebot` para interactuar con la API de Telegram, así como los módulos personalizados `tfg` e `imagenes` para la lógica del chatbot.
	
%	\item Se crea una instancia del bot de Telegram utilizando la API token proporcionada en el archivo `config.py`.
	
%	\item Se definen tres manejadores de mensajes:
%	\begin{itemize}
%		\item El manejador `cmd_start()` se activa cuando un usuario envía el comando `/start` al bot. Este manejador carga una serie de preguntas utilizando la función `cargar_preguntas()` del módulo `tfg` y envía un mensaje de bienvenida al usuario.
		
%		\item El manejador `bot_mensajes_text()` se activa cuando un usuario envía un mensaje de texto al bot. Si el mensaje comienza con %"/", se envía un mensaje de error indicando que el comando no está disponible. Si no, se llama a la función `siguientePregunta()` del módulo `tfg` para procesar la respuesta del usuario y generar una respuesta apropiada.
		
%		\item El manejador `photo()` se activa cuando un usuario envía una foto al bot. El bot descarga la foto, la guarda en el sistema de archivos y llama a la función `analizador_imagenes()` del módulo `imagenes` para analizar la imagen y generar una respuesta apropiada.
%	\end{itemize}
	
%	\item Finalmente, se inicia el bucle de escucha del bot (`bot.infinity_polling()`) para que esté constantemente esperando y respondiendo a los mensajes de los usuarios.
%\end{itemize}



\subsection{Almacenamiento y manejo de la información}

%IDEA AÑADIR SECCIÓN EVALUACIÓN
\subsection{Generación de historias de vida}