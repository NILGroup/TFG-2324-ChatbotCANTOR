% ----------------------------------------------------------------------
%
%                            TFMTesis.tex
%
%----------------------------------------------------------------------
%
% Este fichero contiene el "documento maestro" del documento. Lo único
% que hace es configurar el entorno LaTeX e incluir los ficheros .tex
% que contienen cada sección.
%
%----------------------------------------------------------------------
%
% Los ficheros necesarios para este documento son:
%
%       TeXiS/* : ficheros de la plantilla TeXiS.
%       Cascaras/* : ficheros con las partes del documento que no
%          son capítulos ni apéndices (portada, agradecimientos, etc.)
%       Capitulos/*.tex : capítulos de la tesis
%       Apendices/*.tex: apéndices de la tesis
%       constantes.tex: constantes LaTeX
%       config.tex : configuración de la "compilación" del documento
%       guionado.tex : palabras con guiones
%
% Para la bibliografía, además, se necesitan:
%
%       *.bib : ficheros con la información de las referencias
%
% ---------------------------------------------------------------------

\documentclass[12pt,a4paper,twoside]{book}

%
% Definimos  el   comando  \compilaCapitulo,  que   luego  se  utiliza
% (opcionalmente) en config.tex. Quedaría  mejor si también se definiera
% en  ese fichero,  pero por  el modo  en el  que funciona  eso  no es
% posible. Puedes consultar la documentación de ese fichero para tener
% más  información. Definimos también  \compilaApendice, que  tiene el
% mismo  cometido, pero  que se  utiliza para  compilar  únicamente un
% apéndice.
%
%
% Si  queremos   compilar  solo   una  parte  del   documento  podemos
% especificar mediante  \includeonly{...} qué ficheros  son los únicos
% que queremos  que se incluyan.  Esto  es útil por  ejemplo para sólo
% compilar un capítulo.
%
% El problema es que todos aquellos  ficheros que NO estén en la lista
% NO   se  incluirán...  y   eso  también   afecta  a   ficheros  de
% la plantilla...
%
% Total,  que definimos  una constante  con los  ficheros  que siempre
% vamos a querer compilar  (aquellos relacionados con configuración) y
% luego definimos \compilaCapitulo.
\newcommand{\ficherosBasicosTeXiS}{%
TeXiS/TeXiS_pream,TeXiS/TeXiS_cab,TeXiS/TeXiS_bib,TeXiS/TeXiS_cover%
}
\newcommand{\ficherosBasicosTexto}{%
constantes,guionado,Cascaras/bibliografia,config%
}
\newcommand{\compilaCapitulo}[1]{%
\includeonly{\ficherosBasicosTeXiS,\ficherosBasicosTexto,Capitulos/#1}%
}

\newcommand{\compilaApendice}[1]{%
\includeonly{\ficherosBasicosTeXiS,\ficherosBasicosTexto,Apendices/#1}%
}

%- - - - - - - - - - - - - - - - - - - - - - - - - - - - - - - - - - -
%            Preámbulo del documento. Configuraciones varias
%- - - - - - - - - - - - - - - - - - - - - - - - - - - - - - - - - - -

% Define  el  tipo  de  compilación que  estamos  haciendo.   Contiene
% definiciones  de  constantes que  cambian  el  comportamiento de  la
% compilación. Debe incluirse antes del paquete TeXiS/TeXiS.sty
\include{config}

% Paquete de la plantilla
\usepackage{TeXiS/TeXiS}

% Incluimos el fichero con comandos de constantes
%---------------------------------------------------------------------
%
%                          constantes.tex
%
%---------------------------------------------------------------------
%
% Fichero que  declara nuevos comandos LaTeX  sencillos realizados por
% comodidad en la escritura de determinadas palabras
%
%---------------------------------------------------------------------

%%%%%%%%%%%%%%%%%%%%%%%%%%%%%%%%%%%%%%%%%%%%%%%%%%%%%%%%%%%%%%%%%%%%%%
% Comando: 
%
%       \titulo
%
% Resultado: 
%
% Escribe el título del documento.
%%%%%%%%%%%%%%%%%%%%%%%%%%%%%%%%%%%%%%%%%%%%%%%%%%%%%%%%%%%%%%%%%%%%%%
\def\titulo{\textsc{TeXiS}: Desarrollo de una herramienta basada en lenguaje de apoyo a la terapia basada en reminiscencia}

%%%%%%%%%%%%%%%%%%%%%%%%%%%%%%%%%%%%%%%%%%%%%%%%%%%%%%%%%%%%%%%%%%%%%%
% Comando: 
%
%       \autor
%
% Resultado: 
%
% Escribe el autor del documento.
%%%%%%%%%%%%%%%%%%%%%%%%%%%%%%%%%%%%%%%%%%%%%%%%%%%%%%%%%%%%%%%%%%%%%%
\def\autor{Marta Vicente Navarro}

% Variable local para emacs, para  que encuentre el fichero maestro de
% compilación y funcionen mejor algunas teclas rápidas de AucTeX

%%%
%%% Local Variables:
%%% mode: latex
%%% TeX-master: "tesis.tex"
%%% End:


% Sacamos en el log de la compilación el copyright
%\typeout{Copyright Marco Antonio and Pedro Pablo Gomez Martin}

%
% "Metadatos" para el PDF
%
\usepackage[utf8]{inputenc}
\usepackage{mdframed}
\usepackage{xcolor}
% Definición del estilo Spyder
\definecolor{spyder_keyword}{RGB}{140, 0, 255} % Color de las palabras clave
\definecolor{spyder_comment}{RGB}{63, 127, 95} % Color de los comentarios
\definecolor{spyder_string}{RGB}{206, 145, 120} % Color de las cadenas
\definecolor{spyder_background}{RGB}{255, 255, 255} % Color de fondo del código
\definecolor{spyder_frame}{RGB}{200, 200, 200} % Color del marco del código

\lstdefinestyle{SpyderStyle}{
	language=Python,
	basicstyle=\ttfamily,
	keywordstyle=\color{spyder_keyword},
	commentstyle=\color{spyder_comment},
	stringstyle=\color{spyder_string},
	backgroundcolor=\color{spyder_background},
	frame=single,
	framerule=0.5pt,
	framesep=3pt,
}
\ifpdf\hypersetup{%
    pdftitle = {\titulo},
    pdfsubject = {Plantilla de Tesis},
    pdfkeywords = {Plantilla, LaTeX, tesis, trabajo de
      investigación, trabajo de Master},
    pdfauthor = {\textcopyright\ \autor},
    pdfcreator = {\LaTeX\ con el paquete \flqq hyperref\frqq},
    pdfproducer = {pdfeTeX-0.\the\pdftexversion\pdftexrevision},
    }
    
\fi

%- - - - - - - - - - - - - - - - - - - - - - - - - - - - - - - - - - -
%                        Documento
%- - - - - - - - - - - - - - - - - - - - - - - - - - - - - - - - - - -
\begin{document}

% Incluimos el  fichero de definición de guionado  de algunas palabras
% que LaTeX no ha dividido como debería
\input{guionado}

% Marcamos  el inicio  del  documento para  la  numeración de  páginas
% (usando números romanos para esta primera fase).
\frontmatter
\pagestyle{empty}

\include{Cascaras/cover}
%\include{Cascaras/autorizacion}
% +--------------------------------------------------------------------+
% | Dedication Page (Optional)
% +--------------------------------------------------------------------+

\chapter*{Dedicatoria}

\begin{flushright}
\begin{minipage}[c]{8.5cm}

\flushright{A Mamá, a Mariapi y a los yayos}
\end{minipage}
\end{flushright}
% +--------------------------------------------------------------------+
% | Acknowledgements Page (Optional)                                   |
% +--------------------------------------------------------------------+

\chapter*{Agradecimientos}
%Lo más importante que he aprendido estos años de carrera es, sin duda, a valorar la gente que tengo a mi lado.

Quiero agradecer a todos mis compañeros de clase, en especial al \textit{equipo e} y a los \textit{scape roomers} que han sido lo mejor de la carrera. A Berta, Emma, Nerea y Natalia por ser mi lugar seguro. Si he podido con todo, es porque lo hemos hecho juntas.

Yaya, Yayo, sin la alegría que me dais y la forma en la que me cuidáis no sería la misma. Gracias por darme lo mejor que tengo: mi familia. A la que le dedico este y cualquier logro de mi vida. En especial, a mi tía-madrina-angelito de la guarda por protegerme siempre, darme fuerzas y hacer que sea mucho más feliz. Papá, Inés, Kike, Elvira, Pablo, Edu, Tite y Blanca, gracias.

Pablo, gracias por cuidarme y hacerme sentir en familia. Por creer tanto en mí que me has hecho capaz de lograr cosas que sin tí, nunca hubiera conseguido.

Y por encima de todo quiero darte las gracias a tí, Mamá. Eres la seguridad de que puedo con todo. La certeza de que todo va a estar bien. No sé a qué me dedicaré en el futuro, pero tengo claro lo que quiero: quiero ser como tú. 
















\chapter*{Resumen}
\section*{\tituloPortadaVal}
La enfermedad de Alzheimer es una condición neurodegenerativa que afecta las funciones cognitivas, la memoria, el pensamiento y el comportamiento. Su llegada supone un cambio significativo en la vida de quienes la padecen y en su entorno. Actualmente, se estima que en España 900.000 personas sufren esta y otras formas de demencia, y se proyecta que los casos se duplicarán para el año 2050. Por lo tanto, es de vital importancia desarrollar técnicas que puedan ralentizar el avance de la enfermedad. Aunque no es reversible, existen terapias que pueden mejorar la calidad de vida tanto del paciente como de sus seres queridos.

La terapia de reminiscencia es una modalidad terapéutica que se enfoca en ayudar a las personas a recordar y compartir sus experiencias y recuerdos pasados, especialmente aquellos relacionados con eventos significativos en sus vidas. Aunque es comúnmente empleada con personas mayores, también puede resultar efectiva en otros grupos de edad.

Este proyecto se centra en el desarrollo de un chatbot de apoyo a este tipo de terapia. Para ello, se desarrolla una 

\section*{Palabras clave}
   
\noindent Reminiscencia, Chatbot, historia de vida, alzéhimer, api, procesamiento del lenguaje

   



\begin{otherlanguage}{english}
\chapter*{Abstract}
\section*{\tituloPortadaVal}
Alzheimer's disease is a neurodegenerative condition that affects cognitive functions, memory, thinking, and behavior. Although it is not reversible, there are various treatments to improve the quality of life for those affected. Among them, reminiscence therapy stands out. This non-pharmacological treatment focuses on evoking memories, allowing patients to exercise their minds and give continuity to their life story.

This project aims to develop a tool to support the implementation of these therapies. Specifically, it focuses on developing a chatbot capable of guiding the conversation flow across different topics, extracting the necessary information, and generating specific questions.

To this end, the first stage of the work focuses on conducting an exhaustive study to determine the best tools for its implementation. Specifically, the main language processing APIs and LLMs will be studied. Subsequently, the chatbot is developed iteratively and incrementally.

The final tool is based on the Gemini API, specifically the "gemini-pro" model to implement response analysis and question generation functions. Additionally, the interface developed through Telegram allows for easy and efficient use from various devices (mobile, tablet, or computer).

The code associated with this work can be found in the \href{https://github.com/mavice07/TFG-ChatBotCantor.git}{GitHub repository.}

\section*{Keywords}

\noindent Reminiscence-based therapy, Chatbot, Life history, Alzheimer, API, Gemini, Natural language processing, LLM




\end{otherlanguage}

\ifx\generatoc\undefined
\else
\include{TeXiS/TeXiS_toc}
\fi

% Marcamos el  comienzo de  los capítulos (para  la numeración  de las
% páginas) y ponemos la cabecera normal
\mainmatter

\pagestyle{fancy}
\restauraCabecera

\chapter{Introducción}
\label{cap:introduccion}


\chapterquote{Hay enfermos incurables, pero ninguno incuidable}{Francesc Torralba}


El avance en el campo de la medicina en el último siglo ha permitido un notable aumento en la esperanza de vida a nivel mundial. Sin embargo, este alargamiento de la vida también ha traído consigo un aumento en las enfermedades relacionadas con la vejez, como el Alzheimer. A medida que vivimos más tiempo, enfrentamos un mayor riesgo de desarrollar estas condiciones.


\section{Motivación}

A pesar de los avances en la investigación médica, todavía no se ha encontrado una cura definitiva para el Alzheimer. La mayoría de los tratamientos se centran en aliviar los síntomas y ralentizar la progresión de la enfermedad. Entre estos tratamientos, la terapia de reminiscencia ha surgido como una opción no farmacológica destacada.

La terapia de reminiscencia se centra en estimular los recuerdos del pasado del paciente, lo que puede tener beneficios significativos en su bienestar social, mental y emocional. Ayuda a las personas a recordar y compartir experiencias pasadas, lo que puede ser reconfortante y estimulante, especialmente para aquellos que enfrentan el desafío de la pérdida de memoria asociada con el Alzheimer.

Durante el período académico 2023-2024, se desarrolló la aplicación YayoBot con el objetivo de facilitar a los terapeutas la realización de terapias basadas en reminiscencia, simplificando y agilizando el proceso. Este desarrollo parte desde cero y busca crear un chatbot funcional y útil, capaz de mantener una conversación útil con el paciente para extraer la información necesaria.

Este Trabajo de Fin de Grado tiene como propósito asistir a terapeutas, familiares o amigos de pacientes con demencia en la obtención del material necesario para llevar a cabo terapias de reminiscencia. Se busca mejorar la eficacia de estas terapias y, en consecuencia, la calidad de vida tanto de los pacientes como de sus familiares. 


\section{Objetivos}
Este trabajo  busca desarrollar un chatbot que permita a los terapeutas aplicar terapia de reminiscencia a sus pacientes. En concreto, se centra en la primera etapa de búsqueda de la información con el objetivo de, más adelante generar las historias de vida y aplicar la terapia. 

Para cumplir este objetivo general, se abordan los siguientes objetivos específicos:

 \begin{itemize}
 	
\item Desarrollar un primer chatbot básico capaz de realizar preguntas predefinidas y almacenar las respuestas de forma eficiente.
 
 \item Estudiar las técnicas modernas de procesamiento del lenguaje como los LLMs, bibliotecas como NLTK y spaCy. Además, de las diferentes APIs de para el desarrollo de chatbots. 
 
\item Mejorar el primer chatbot haciendolo más inteligente y capaz de analizar las respuestas, identificar la información omitida y hacer preguntas específicas para obtener la información faltante.
 
\item Hacer una versión final del chatbot que sea capaz de analizar las repuestas y sepa tirar del hilo. También que sea capaz de generar preguntas adecuadas. 

\item Generar una interfaz sencilla de usar para que haga del chatbot una herramienta útil para el próposito para el que ha sido desarrollada.

\item Ordenar la información obtenida para que sea fácil de analizar, entender y procesar, con la intención de generar historias de vida a partir de ella. 

\end{itemize}

Para llevar el control de versiones utilizaremos el repositorio de github:\\
 https://github.com/NILGroup/TFG-2324-ChatbotCANTOR. 

\section{Estructura de la memoria}
El presente documento esta formado por 5  capítulos, incluyendo el presente capítulo. Estos se organizan como sigue. 
\begin{itemize}
	\item \textbf{Capítulo 1: Introducción}. El capítulo presente, se presenta el proyecto, motivación, objetivos y estructura del mismo. 
	\item \textbf{Capítulo 2: Estado de la cuestión}. Es segundo capítulo, nos muestra una aproximación a las historias de vida, la enfermedad del Alzhéimer y la terapia de remiscencia. También muestra otros trabajos relacionados, contexto y precedentes del proyecto. 
	\item \textbf{Capítulo 3: Marco teórico} En este tercer capítulo se explica el estudio profundo de las herramientas de procesamiento del lenguaje, interfaces y almacenamiento de la información que se llevo a cabo para decidir cómo implementar el chatbot. Así se explican las diferentes opciones y los motivos que llevaron a elegir cada una de ellas. 
	\item \textbf{Capítulo 4: Desarrollo del chatbot}
	Este capítulo cuenta cómo se ha ido desarrollando el proyecto, explicando cada una de las versiones que se han llevado a cabo. Se explica, desde la primera versión, los problemas que se han ido encontrando, las funcionalidades extra añadidas etc. 
	\item \textbf{Capítulo 5: Versión final y resultados}
	En el capítulo 5, se presenta la interfaz final con todas sus funcionalidades. El software y las herramientas utilizados, la guía para la instalación y puesta en marcha, la arquitectura del sistema final etc. 
	\item \textbf{Anexos} Adicionalmente, en el anexo se muestra una conversación con un paciente real para ver el funcionamiento del chatbot. 
	
\end{itemize}




\begin{otherlanguage}{english}
	\chapter*{Introduction}
\label{chap:introduction}

\chapterquote{There are incurable patients, but none uncareable}{Francesc Torralba}

The constant increase in life expectancy worldwide has led to an increasingly aging population. This has brought with it an increase in diseases associated with old age, such as Alzheimer's and other types of dementia. Research efforts in these fields have focused on developing treatments that can slow the progression of this disease. Memory plays a fundamental role in shaping our identity through our experiences. Its loss can be devastating as it profoundly affects the quality of life of the person and those around them.

Among non-pharmacological strategies for treating cognitive decline and memory loss, reminiscence therapy stands out. This technique aims to stimulate the patient's past memories, which can have significant benefits for their social, mental, and emotional well-being.

On the other hand, artificial intelligence is increasingly present in our daily lives. It has a wide variety of uses and applications ranging from virtual assistants on our smartphones to recommendation systems on streaming platforms and e-commerce. Additionally, artificial intelligence is used in medicine for more accurate diagnoses, in industry to automate manufacturing processes, and even in autonomous vehicle driving. This omnipresence of artificial intelligence is profoundly transforming various sectors, driving efficiency, service personalization, and generating new technological development opportunities.

This project aims to find a way to apply artificial intelligence, specifically natural language processing, to reminiscence therapy. Specifically, it seeks to develop a conversational support tool for conducting this type of therapy.

This chapter has three main objectives. First, to explain the current motivation behind this problem. Second, to clarify the goals set as the project's starting point. Finally, with the intention of providing an overview of the project, there will be a final section explaining the structure of this document, emphasizing what can be found in each chapter.

\section{Motivation}

Despite advances in medical research, no definitive cure for Alzheimer's has yet been found. Most treatments focus on alleviating symptoms and slowing the progression of the disease. Among these treatments, reminiscence therapy has emerged as a prominent non-pharmacological option.

This therapy focuses on stimulating the patient's past memories, which can have significant benefits for their social, mental, and emotional well-being. It helps people recall and share past experiences, which can be comforting and stimulating, especially for those facing the challenge of memory loss associated with Alzheimer's.

During the 2023-2024 academic year, a chatbot was developed with the aim of facilitating therapists in conducting reminiscence-based therapies, simplifying and streamlining the process. This development starts from scratch and seeks to create a functional and useful chatbot capable of maintaining a useful conversation with the patient to extract the necessary information.

This Final Degree Project aims to assist therapists, family members, or friends of patients with dementia in obtaining the necessary material for conducting reminiscence therapies. It seeks to improve the effectiveness of these therapies and, consequently, the quality of life for both patients and their families.

\subsection{Alzheimer's Disease}
Alzheimer's disease (AD) is the leading cause of dementia in older adults. AD is a complex disease determined by multiple factors. Sometimes it is hereditary. It is characterized by the loss of neurons and synapses, along with the presence of amyloid plaques and neurofibrillary degeneration. Clinically, it manifests as progressive dementia, starting with recent memory failures and advancing to total dependency.

According to \cite{Donoso2003}, the incidence of AD increases with age, being rare before the age of 50 and affecting 1-2$\%$ of individuals at age 60, 3-5$\%$ at age 70, 15-20$\%$ at age 80, and up to half of those over 85. It is more frequent in women, possibly due to their greater longevity.

The disease develops in different stages. In the initial stage, symptoms may go unnoticed or be attributed to simple forgetfulness. The affected person may experience difficulties remembering names, recent events, or finding the right words in conversations. Despite these challenges, they generally retain the ability to perform daily tasks with some independence. However, they may begin to lose interest in previously enjoyed activities.

As the disease progresses, the symptoms become more evident and problematic. Memory loss becomes more pronounced, with difficulties recognizing close family and friends. Additionally, problems with time and space orientation may arise, leading to disorientation even in familiar environments. Communication skills are also affected, with difficulties following conversations or expressing thoughts coherently.

Finally, Alzheimer's reaches its most devastating stage. Memory loss is profound and complete, with an inability to recall even recent events or recognize familiar faces. The affected person may experience significant changes in personality and behavior, becoming agitated, anxious, or even aggressive at times. The ability to perform basic daily life activities, such as dressing or feeding, is severely compromised, and constant supervision becomes essential.

The pharmacological treatment of Alzheimer's disease (AD) is based on the use of medications to improve cognitive defects and correct behavioral disorders. The most valued drugs are acetylcholinesterase inhibitors, which attempt to compensate for the loss of cholinergic neurons in the cerebral cortex. Treatment also includes drugs to treat behavioral disorders, such as antidepressants, tranquilizers, or sleep inducers (\cite{chung2000neurobehavioral}).

In addition to pharmacological treatment, physical and mental activity is crucial for stimulating brain activity and preventing memory loss. Family guidance is essential to help family members manage behavioral disorders and increase the quality of life for AD patients. An example of a non-pharmacological therapy that helps maintain mental activity is reminiscence therapy.

\subsection{Reminiscence Therapy}
Reminiscence, according to the definition by CEAFA (Spanish Confederation of Associations of Relatives of People with Alzheimer's and Other Dementias), is a technique that seeks to evoke memories in people, especially those related to significant events in their lives.

Reminiscence therapies use various materials such as photographs, videos, newspaper clippings, audios, and significant objects to stimulate memory and activate it through the emotions these elements evoke in the patient.

In addition to stimulating the five senses to trigger memory, storytelling known to the patient is also used. Therefore, it is crucial to know the individual's past experiences to tailor the materials used in the therapies. The participation of people close to the patient, such as family members and caregivers, in reminiscence exercises can significantly improve their quality of life.

It is recommended to build stories based on the patient's life experiences for these memory exercises. Reminiscence therapies are considered part of a broader process in which the individual attempts to recover and link memories spanning much of their life.

A review on the potential effects of reminiscence therapy on people with dementia and their caregivers concluded that it decreased behavioral disturbances and depressive symptoms, and improved cognition (\cite{huang2015reminiscence}). Another review observed improvements in mood, cognitive ability, social behavior, and general well-being (\cite{cotelli2012reminiscence}), and recently, it was found that reminiscence therapy improved physical health and increased patient participation (\cite{irazoki2027eficacia}).

\subsection{Life Stories}
Life Stories are detailed records of the most relevant aspects of a patient's life or significant people in their life. They are especially important in the early stages of Alzheimer's when memory is still relatively intact. These stories provide dignity to the patient and allow those around them to better understand their identity as memory losses become significant.

Creating these stories is vital for preserving the patient's identity. They can include details about family, professional career, hobbies, travels, and other important aspects of the individual's life. They can be created in various forms, such as writing a book, creating photo collages, producing a movie, or using a "memory box."

It is essential that these stories reflect the patient's personal perspective, including emotions, feelings, and interpretations. It is not simply about recounting chronological facts but capturing the unique essence of the person.

A therapist usually undertakes the task of collecting and structuring the significant events of the patient's life to create the Life Story. The structure can vary depending on the patient's needs and therapeutic goals.

The involvement of family members or acquaintances can facilitate memory recall and enrich the Life Story. Additionally, it strengthens communication between the patient and their loved ones, facilitating the therapeutic process.

\section{Objectives and Work Methodology}
\label{sec:objectives}

The objective of this work is to develop a chatbot that supports the conduction of reminiscence therapy. Specifically, the chatbot should meet the following requirements:

\begin{itemize}
	\item Be capable of maintaining a conversation, guiding the flow through predefined topics about which data from the patient is sought.
	\item Extract the desired information from the patient's responses, identify additional and missing information.
	\item Generate questions related to the responses, to obtain any initially missing information.
	\item Have a user-friendly interface, making this chatbot a useful tool.
\end{itemize}


To achieve these objectives, the following methodology will be followed:

\begin{itemize}
	\item Study Alzheimer's disease and reminiscence therapy to understand the characteristics a conversational bot must have to be helpful in this area.
	\item Conduct an in-depth study of various natural language processing tools, from libraries like NLTK and spaCy to different APIs and LLM models. This will allow us to choose the most appropriate tools for subsequent development.
	\item Once the development phase begins, the first step will be to implement a basic chatbot capable of asking predefined questions and efficiently storing responses.
	\item Improve the tool, making it smarter and capable of analyzing responses, identifying omitted information, and asking specific questions to obtain the missing information.
	\item Develop a final version capable of analyzing responses and probing further, in addition to the functionalities of previous prototypes.
	\item Manage the storage of extracted information.
	\item Create a simple and usable interface, making the chatbot a useful tool for its intended purpose.
	\item Allow the user to accompany their responses with images, both to help them give more complete answers and to obtain additional information through processing them.
\end{itemize}

To manage version control, we will use the GitHub repository: \\
https://github.com/NILGroup/TFG-2324-ChatbotCANTOR.

\section{Structure of the Document}
This document consists of 6 chapters, including the current one. They are organized as follows:
\begin{itemize}
	\item \textbf{Chapter 1: Introduction}. The current chapter presents the project, motivation, objectives, and structure.
	\item \textbf{Chapter 2: State of the Art}. This chapter provides an overview of life stories, Alzheimer's disease, and reminiscence therapy. It also presents related works, context, and precedents of the project.
	\item \textbf{Chapter 3: Technologies Used}. This third chapter explains the in-depth study of natural language processing tools, interfaces, and information storage conducted to decide how to implement the chatbot. It explains the different options and the reasons for choosing each.
	\item \textbf{Chapter 4: Prototype Development}. This chapter explains how the project has been developed, describing each version carried out. It details the initial version, the problems encountered, additional functionalities added, etc.
	\item \textbf{Chapter 5: Designing a Reminiscence Therapy Support Chatbot}. In Chapter 5, the final tool with all its functionalities is presented. The software and tools used, the guide for installation and setup, the architecture of the final system, etc.
	\item \textbf{Chapter 6: Conclusions and Future Work}. Chapter 6 discusses the conclusions reached during the project development and the work that could be done in the future to expand it.
\end{itemize}
\end{otherlanguage}
\chapter{Estado de la Cuestión}
\label{cap:estadoDeLaCuestion}
Este capítulo muestra la perspectiva teórica de la investigación llevada a cabo como trabajo previo al desarrollo del código. El conocimiento que se presenta en las siguientes páginas es necesario para entender cuál era el contexto del problema y el porqué de las decisiones que se han tomado para la construcción de la solución.

La estructura del capítulo muestra, en orden temporal, las herramientas del procesamiento de lenguaje que se han ido popularizando, desde las alternativas históricas hasta la situación actual. De está forma, nos permite entender su importancia y conocer otras posibles implementaciones.

Finalmente, este capítulo muestra trabajos relacionados, ya sean puntos de partida para el trabajo actual, trabajos que complementan al presente proyecto, o incluso otros trabajos que cuyo estudio ha servido como herramienta de aprendizaje de cara a preparar el $chatbot$. 

En conclusión, se pretende dar una perspectiva global de la situación actual en la que se encuentra el procesamiento del lenguaje en general y el desarrollo del $chatbot$ en particular.  
\section{Evolución del Procesamiento del Lenguaje Natural}
El procesamiento del lenguaje natural (PLN) ha experimentado una evolución notable, impulsada por avances tecnológicos como la inteligencia artificial (IA). La integración de técnicas de IA en el PLN dió lugar a la rama del procesamiento natural del lenguaje (PLN), que supuso todo un hito en el ámbito del lenguaje escrito.

Inicialmente, los algoritmos de PLN se basaban en reglas, pero con el tiempo se adoptaron modelos de clasificación supervisada. Sin embargo, estos modelos enfrentaban limitaciones al no capturar el contexto completo de las palabras en una frase. Tres avances clave marcaron el camino hacia una PNL más avanzada. 

Primero, el surgimiendo de los modelos $word$ $embeddings$ descritos en la sección \ref{sec:WordEmbeddings}, desarrollados en 2013 que permiten representar palabras en un espacio vectorial considerando su contexto, lo que facilita la comprensión de sinónimos y relaciones entre palabras. 

En segundo lugar, la arquitectura de redes neuronales profundas conocida como $transformers$, descritos en la sección \ref{sec:Transformers}, revolucionó el campo al capturar el contexto completo de un texto mediante una matriz de atención. Además, los modelos basados en $transformers$ permiten la transferencia de aprendizaje, lo que facilita la adaptación a diversas aplicaciones.

Finalmente, el surgimiento de modelos generativos de lenguaje multipropósito de gran tamaño, como el archiconocido ChatGPT, ha llevado la PNL a nuevos horizontes. Estos modelos, que se describen en la sección \ref{sec:LLM} con cientos de millones de parámetros, pueden generar texto de calidad comparable a la humana y están generando debates sobre su impacto en diversos ámbitos. 

A pesar de estos avances, persisten desafíos en el PLN, como la necesidad de bases de datos específicas y validadas para el entrenamiento de modelos especializados. Los investigadores son llamados a trabajar en la construcción de nuevas bases de datos y en el desarrollo de esquemas de preprocesamiento y ajuste, con el objetivo de impulsar soluciones a problemas específicos a nivel global.

\section{Word embeddings}
\label{sec:WordEmbeddings}

Los $word$ $embeddings$ \citep{wordEmbeddings} son una técnica importante en el procesamiento del lenguaje natural que consiste en representar palabras y documentos como vectores numéricos en un espacio de dimensiones reales. Este enfoque permite que palabras con significados similares tengan representaciones vectoriales similares, lo que facilita que las computadoras comprendan el contenido basado en texto de manera más efectiva.

En los $word$ $embeddings$, las palabras se representan como vectores numéricos en un espacio dimensional reducido, lo que permite capturar información semántica y sintáctica entre palabras. Estos vectores se utilizan como características para alimentar modelos de aprendizaje automático, lo que permite trabajar con datos de texto y preservar la información semántica y sintáctica. Existen diferentes técnicas para tratar los $word$ $embeddings$.

\subsection{TF-IDF}

La frecuencia de término-inversa de frecuencia de documento (TF-IDF) es un algoritmo de aprendizaje automático que se utiliza para la incrustación de palabras en texto. Consta de dos métricas: frecuencia de término (TF) y la inversa de la frecuencia de documento (IDF).

Este algoritmo trabaja en una medida estadística para encontrar la relevancia de las palabras en el texto, que puede estar en forma de un solo documento o varios documentos referidos como corpus.

\begin{center}
$\textbf{tf-idf}_{i,j} =$ Frecuencia del término $i$ en el documento $j$ $\times$ Frecuencia inversa de documentos del término $i$
\end{center}

El puntaje de frecuencia de término (TF) mide la frecuencia de las palabras en un documento particular. En otras palabras, cuenta la ocurrencia de palabras en los documentos.
\begin{center}
	$\textbf{tf}_{i,j} =  \frac{\textit{Frecuencia del término } i \textit{ en el documento } j}{\textit{Número total de términos en } j}$ 
\end{center}


La inversa de la frecuencia de documento (IDF) mide la rareza de las palabras en el texto. Se le otorga más importancia que el puntaje de frecuencia de término porque, aunque el puntaje de TF otorga más peso a las palabras que ocurren con frecuencia, el puntaje de IDF se centra en las palabras raramente utilizadas en el corpus que pueden contener información significativa.
\begin{center}
	$\textbf{idf}_{i} = \log \left( \frac{\textit{Número total de documentos}}{\textit{Número de documentos que contienen el término } i} \right)$
\end{center}


El algoritmo TF-IDF se utiliza en tareas básicas de procesamiento de lenguaje natural y aprendizaje automático, como la recuperación de información, eliminación de palabras vacías, extracción de palabras clave y análisis de texto. Sin embargo, no captura eficientemente el significado semántico de las palabras en una secuencia.

\subsection{Bolsa de palabras}
	
El Bag of Words (BoW) es una técnica de procesamiento de texto ampliamente utilizada en el procesamiento del lenguaje natural (PLN) y la minería de texto. Esta técnica consiste en representar un documento de texto como un conjunto de palabras, ignorando el orden y la estructura gramatical.
	
En el modelo BoW, se crea un vocabulario de todas las palabras únicas en un conjunto de documentos de texto. Cada documento se representa como un vector de tamaño igual al tamaño del vocabulario, donde cada posición en el vector indica la frecuencia de una palabra en el documento.
	
Por ejemplo, si el vocabulario contiene las palabras "gato", "perro" y "juguete", y un documento tiene una frecuencia de dos para "gato" y tres para "perro", su vector de representación BoW sería [2,3,0].
	
La técnica BoW es útil para la clasificación y agrupación de documentos basados en su contenido textual. Se utiliza en aplicaciones como análisis de sentimientos, clasificación de texto y recomendación de contenidos.
	
El proceso de creación de la matriz BoW se lleva a cabo en varios pasos:
\begin{enumerate}
	\item \textbf{Creación del vocabulario}: Se genera un conjunto de palabras únicas a partir de todos los documentos de texto.
	\item \textbf{Vectorización del texto}: Cada documento se convierte en un vector de tamaño igual al vocabulario, donde las palabras presentes en el documento tienen un valor de uno, y las ausentes tienen un valor de cero.
	\item \textbf{Normalización del peso}: Los vectores se normalizan dividiendo cada valor por la suma total de valores en el vector.
\end{enumerate}

Una vez creada la matriz BoW, se puede utilizar en diversas aplicaciones de inteligencia artificial, como la clasificación de textos, el análisis de sentimientos y la agrupación de documentos.


\subsection{Word2Vec}
El método Word2Vec, desarrollado por Google en 2013, revolucionó el campo del procesamiento del lenguaje natural (PLN) al ofrecer una forma innovadora de entrenar incrustaciones de palabras. Este enfoque se basa en una idea distributiva que utiliza skip-grams o una técnica llamada Bolsa Continua de Palabras (CBOW).

En esencia, Word2Vec emplea redes neuronales poco profundas con capas de entrada, salida y proyección. Estas redes tienen como objetivo reconstruir el contexto lingüístico de las palabras, considerando tanto su orden en el texto como su contexto futuro.

El proceso implica iterar sobre un corpus de texto para aprender las asociaciones entre palabras. Se fundamenta en la premisa de que las palabras que aparecen juntas en un texto tienen una similitud semántica entre ellas. Esto permite asignar representaciones vectoriales a las palabras que son cercanas geométricamente en el espacio vectorial.

La similitud del coseno se utiliza como métrica para medir cuán similares son dos palabras o documentos en su significado. Si el ángulo entre los vectores de palabras es pequeño, la similitud del coseno es alta, lo que indica que las palabras tienen significados similares. Por el contrario, si el ángulo es de 90 grados, la similitud es baja, lo que indica que las palabras son independientes en su contexto.

\section{Transformers}
\label{sec:Transformers}

Los modelos pre-entrenados son modelos de aprendizaje profundo o $Deep Learning$ que sirven para realizar diversas tareas de Procesamiento de Lenguaje. Son pre-entrenados bajo grandes conjuntos de datos, lo que les permite ajustarse a atareas específicas sin requerir un entrenamiento desde cero. Este pre-entrenamiento es clave para entender por qué son tan valiosos, ya que permiten construir un sistema de generación de lenguaje sin un gran esfuerzo computacional (normalmente los entrenamientos tardan semanas o meses incluso con los mejores computadores). 

Dentro de los modelos pre-entrenados, toman especial importancia los $transformers$ (\cite{transformers}). Desde el primer momento, revolucionaron el PLN al introducir mecanismos de atención auto-ajustable, paralelización eficiente, captura de contexto bidireccional y pre-entrenamiento masivo. Todo ello permite construir modelos más poderosos y efectivos. 

\subsection{Aprendizaje por transferencia}
Los modelos pre-entrenados, frente a los modelos estadísticos u otros tipos de algoritmos de \textit{Deep Learning}, suponen grandes ventajas debido al pre-entrenamiento bajo un gran conjunto de datos y un pequeño ajuste posterior a realizar para adaptarlo a la tarea que se desee resolver, exigiendo mucho menos coste computacional y requiriendo para ello menos tiempo y esfuerzo. Todo esto es posible gracias al aprendizaje por transferencia o \textit{Transfer Learning}.

El \textit{Transfer Learning} es fundamental en el desarrollo y la eficiencia de los modelos pre-entrenados. Aparecen como solución al paradigma del aprendizaje aislado que presentan algoritmos como el aprendizaje supervisado tradicional. Estos modelos pueden aprovechar el conocimiento adquirido previamente en grandes conjuntos de datos para adaptarse a nuevas tareas con un costo computacional y de tiempo significativamente menor que si se entrenaran desde cero. Este enfoque es especialmente valioso en campos como la visión artificial o el PLN.

En el ámbito de la visión artificial, los modelos pre-entrenados pueden ser adaptados para tareas específicas, como la clasificación de imágenes, mediante el aprendizaje por transferencia. Por ejemplo en \cite{tu2018transfer}, el sistema utiliza redes neuronales convolucionales pre-entrenadas para reconocer perros en imágenes, aprovechando el conocimiento previo adquirido por el modelo.

En PLN, los modelos pre-entrenados también son esenciales para resolver diversas tareas, como la detección de noticias falsas (\cite{slovikovskaya2019transfer}). Estos modelos pueden emplear el aprendizaje por transferencia para adaptarse a nuevas tareas, como la clasificación de texto, utilizando el conocimiento previo obtenido durante el entrenamiento inicial.

Dentro del aprendizaje por transferencia, existen varias técnicas, como la adaptación de dominio, la confusión de dominio, el aprendizaje de una sola muestra ($One-shot Learning$), el aprendizaje de cero muestras ($Zero-shot Learning$) y el aprendizaje multitarea. Este último, utilizado por modelos como T5, tiene como objetivo crear modelos generalistas capaces de resolver múltiples tareas, en contraposición a modelos especializados en una sola tarea.

\section{Modelos de Lenguaje de Gran Tamaño (LLM)}
\label{sec:LLM}

La invención de los transformadores marcó el comienzo de la era de los grandes modelos de lenguaje modernos. Desde 2018, los laboratorios de IA han comenzado a entrenar modelos cada vez más grandes, conocidos como LLM.
	
Los Modelos de Lenguaje de Gran Tamaño (LLM) son modelos de aprendizaje profundo que se entrenan con grandes cantidades de datos utilizando la arquitectura de transformadores. Estos modelos constan de un codificador y un decodificador que trabajan en conjunto para entender y generar texto. 

Hay tres lineas de desarrollo principales para estos modelos de lenguaje, y múltitud de modelos como se puee ver en la figura \ref{fig:3.1}. 

\begin{figure}[h]
	\centering
	\includegraphics[width=1\textwidth]{Imagenes/treeLLM}
	\caption{Desarrollo de los LLM por\cite{yang2023harnessing}}
	\label{fig:3.1}
\end{figure}

Por un lado, el grupo ``solo codificador'', mostrado en rosa en la figura \ref{fig:3.1} incluye LLM que son buenos para la comprensión del texto porque permiten que la información fluya en ambas direcciones del texto. En azul, podemos ver el grupo ``solo decodificador'' que incluye LLM que son buenos en la generación de texto porque la información solo fluye de izquierda a derecha del texto para generar nuevas palabras de manera eficiente y autorregresiva. Finalmente, hay un tipo codificador-decodificador (mostrado en verde) que combina ambos aspectos y se usa para tareas que requieren comprender una entrada y generar una salida, como la traducción. 

Dentro de este último grupo encontramos la mayoría de los modelos que fueron considerados para el desarrollo de este trabajo como los distintos modelos de GPT, los de Google (Bard, que evolucionó a Gemini, LaMDA o PaLM).
\subsection{BERT}
Los modelos de lenguaje enmascarados, como los Masked Language Models, enmascaran un cierto porcentaje de palabras en una oración y se espera que el modelo prediga esas palabras en función del contexto restante \cite{rothman2022}. Un ejemplo representativo de este enfoque es BERT.

BERT (\textit{Bidirectional Encoder Representations from Transformers}) es un modelo de lenguaje enmascarado basado en la arquitectura Transformer \cite{devlin2019bert}. Desarrollado por Google en 2018, BERT ha demostrado un rendimiento sobresaliente en una variedad de tareas de procesamiento de lenguaje natural.

BERT se pre-entrena en dos grandes corpus de texto en inglés sin etiquetar: BookCorpus \cite{zhu2015aligning} y Wikipedia. Estos corpus garantizan una amplia cobertura de datos y una buena calidad para el entrenamiento del modelo.

Aunque inicialmente no estaba diseñado para la generación de texto, BERT ha sido adaptado para esta tarea, logrando mejoras significativas en comparación con modelos como GPT-2 \cite{wang2019bert}.

Existen numerosas variantes de BERT, algunas pre-entrenadas en dominios específicos y otras con un ajuste fino para tareas específicas \cite{rajasekharan2019review}. Por ejemplo, Beto es la versión en español de BERT, entrenada en un corpus extenso en dicho idioma \cite{canete2020spanish}.

La innovación clave de BERT es su enfoque bidireccional, que permite al modelo comprender el contexto de una palabra en función de su entorno completo. A diferencia de modelos unidireccionales como GPT-2, BERT considera tanto el contexto anterior como el posterior a una palabra dada.

La arquitectura de BERT consiste en una pila de codificadores de transformadores, con un número variable de capas dependiendo de la versión del modelo. Este enfoque permite a BERT realizar múltiples tareas de procesamiento de lenguaje, incluyendo Modelado de Lenguaje Enmascarado y Predicción de la Siguiente Oración \cite{devlin2019bert}.		


\subsection{T5}

T5, o Text-to-Text Transfer Transformer \cite{T5}, es un marco unificado para el aprendizaje por transferencia en procesamiento de lenguaje natural (PLN). Este enfoque revolucionario convierte todos los problemas basados en texto en un formato de texto a texto, donde el modelo recibe texto como entrada y genera nuevo texto como salida. Inspirado en marcos anteriores para tareas de NLP, como la modelización de lenguaje o la extracción de fragmentos, T5 permite aplicar el mismo modelo, objetivo de entrenamiento, procedimiento de entrenamiento y proceso de decodificación a cada tarea considerada.

El objetivo principal de T5 es proporcionar una perspectiva exhaustiva sobre el estado actual del campo de la transferencia de aprendizaje para PLN. En lugar de proponer nuevos métodos, se enfoca en la exploración y comparación empírica de técnicas existentes. Además, para facilitar futuros trabajos en este campo, se liberan conjuntos de datos, modelos pre-entrenados y código fuente.

\begin{figure}[h]
	\centering
	\includegraphics[width=0.9\textwidth]{Imagenes/T5}
	\caption{Diagrama del modelo T5 de \cite{T5}}
	\label{fig:3.2}
\end{figure}

Cómo se puede ver en la figura \ref{fig:3.2} el modelo T5 nos permite realizar diversas tareas como la traducción o la realización de resumenes.
%completar con las otras tareas que nos permite hacer no identifico que hace en lo rojo y amarillo 

\subsection{GPT (Generative Pretrained Transformer)}

Los modelos de lenguaje informales, también conocidos como modelos de lenguaje generativos, se centran en predecir palabras o tokens enmascarados dentro de una oración. Visualicemos un token enmascarado como un espacio en blanco dentro de una frase. En este escenario, el modelo solo considera el contexto anterior (palabras a la izquierda) y no tiene en cuenta el contexto posterior. La dirección del procesamiento no es crucial, siempre y cuando siga una dirección única, utilizando sólo las palabras relevantes del contexto para hacer predicciones, mientras descarta el resto. Esta propiedad clave de los modelos se refleja en su esquema de entrenamiento \citep{rothman2022}.

\subsubsection{GPT-1}

GPT-1 (Generative Pre-trained Transformer 1) fue el primer gran modelo de lenguaje desarrollado por OpenAI tras la introducción de la arquitectura Transformer por Google en 2017. En junio de 2018, OpenAI publicó un artículo titulado ``Improving Language Understanding by Generative Pre-Training'' \citep{radford2018improving}, donde presentaron este modelo inicial junto con el concepto general de un Transformer generativo pre-entrenado.

Hasta ese momento, los modelos de procesamiento de lenguaje neuronal con mejor desempeño se basaban principalmente en el aprendizaje supervisado a partir de grandes conjuntos de datos etiquetados manualmente. Esta dependencia limitaba su utilidad en conjuntos de datos poco etiquetados y hacía que el entrenamiento de modelos extremadamente grandes fuera costoso y lento. Muchos idiomas (como el suajili o el criollo haitiano) resultaban difíciles de abordar debido a la escasez de datos disponibles para la construcción de corpus. En contraste, el enfoque ``semi-supervisado'' de GPT implicaba dos etapas: una etapa de pre-entrenamiento generativo no supervisado, donde se establecían los parámetros iniciales mediante un objetivo de modelado de lenguaje, y una etapa de ajuste fino discriminatorio supervisado, donde estos parámetros se adaptaban a una tarea específica.

La elección de la arquitectura Transformer, en lugar de técnicas anteriores basadas en redes neuronales con atención mejorada, dotó a los modelos GPT de una memoria más estructurada que la alcanzada mediante mecanismos recurrentes, lo que resultó en un sólido desempeño en transferencia a diversas tareas.

\subsubsection{GPT-2}

GPT-2 es uno de los modelos más emblemáticos de lenguaje casual que sigue la arquitectura Transformer basada en auto-atención (masked self-attention). Presentado por OpenAI en 2019, fue aclamado desde el principio en el campo del Procesamiento de Lenguaje debido a su gran escala, con más de 1.5 billones de parámetros.

El objetivo principal de este sistema es construir una distribución de probabilidad donde cada posible palabra a generar recibe una probabilidad en función del contexto anterior. Este modelo fue pre-entrenado en un gran corpus de texto inglés utilizando el método auto-supervisado (self-supervised). Su propósito fundamental es la predicción de la siguiente palabra en una secuencia de palabras u oraciones.

Para el pre-entrenamiento, se creó un conjunto de datos llamado WebText, obtenido al extraer millones de páginas web a partir de enlaces de salida de Reddit que cumplían con ciertos criterios de calidad. Las páginas de Wikipedia asociadas a estos enlaces fueron excluidas. El resultado fue un corpus masivo de 40GB de textos adaptados para el entrenamiento de este modelo \citep{radford2019language}.

La estructura de GPT-2 se asemeja a la del Transformer original. Inicialmente, el modelo constaba de un codificador y un decodificador diseñados para tareas específicas como traducción automática. Sin embargo, GPT-2 abandonó esta arquitectura convencional y optó por una serie exclusiva de decodificadores basados en el Transformer. El número de decodificadores utilizados varía según el tamaño de GPT-2: desde doce en la versión Small hasta cuarenta y ocho en la versión Extra Large. Esta configuración basada únicamente en decodificadores es una característica distintiva de GPT-2, ilustrada en la Figura \ref{fig:gpt2_architecture}.

\begin{figure}
	\centering
	\includegraphics[scale=0.5]{Imagenes/gpt2_architecture}
	\caption{Arquitectura de GPT-2}
	\label{fig:gpt2_architecture}
\end{figure}

\subsubsection{GPT-3}

GPT-3 aumentó significativamente la cantidad de parámetros en comparación con GPT-2, alcanzando hasta 175 mil millones, lo que permitió una mayor complejidad y capacidad de aprendizaje. Gracias a su mayor capacidad y tamaño de conjunto de datos, GPT-3 demostró una mejor habilidad para generalizar y comprender una variedad más amplia de contextos y tareas. Además, requiere menos entrenamiento adicional para tareas específicas en comparación con GPT-2.

Por otro lado, a pesar de su mayor capacidad, GPT-3 logró reducir la generación de texto tóxico en comparación con GPT-2, aunque todavía se necesitaron estrategias de mitigación.

GPT-3 produce textos con mayor precisión y coherencia, lo que resulta en una calidad general de generación de texto más alta y una capacidad para realizar tareas más complejas.

\subsubsection{GPT-4}

GPT-4,\citep{GPT4} es el siguiente modelo de gran escala de OpenAI capaz de aceptar entradas de imagen y texto para producir salidas de texto. Aunque menos capaz que los humanos en muchos escenarios del mundo real, GPT-4 exhibe un rendimiento a nivel humano en diversos puntos de referencia profesionales y académicos, incluida la aprobación de un examen simulado de abogacía con una puntuación aproximadamente un 10$\%$ superior a la media de los examinados humanos. GPT-4 es un modelo basado en Transformers preentrenado para predecir el siguiente token en un documento. El proceso de ajuste posterior al entrenamiento mejora su rendimiento en medidas de factualidad y adherencia al comportamiento deseado.

En una serie de benchmarks tradicionales de Procesamiento de Lenguaje Natural (NLP), GPT-4 supera tanto a modelos de lenguaje grandes anteriores como a la mayoría de los sistemas de vanguardia (que a menudo requieren entrenamiento específico para el benchmark o ingeniería manual). En el benchmark MMLU, que cubre 57 temas en inglés, GPT-4 no solo supera a los modelos existentes en inglés, sino que también demuestra un rendimiento sólido en otros idiomas.

A pesar de sus capacidades, GPT-4 tiene limitaciones similares a modelos anteriores, como la falta de fiabilidad completa, una ventana de contexto limitada y la incapacidad de aprender de la experiencia. Las capacidades y limitaciones de GPT-4 plantean desafíos de seguridad significativos y novedosos..

\subsection{Llama}
A diferencia de la creencia común de que más parámetros conducen a un mejor rendimiento, investigaciones recientes muestran que modelos más pequeños entrenados con más datos pueden superar a los modelos más grandes. Se ha desarrollado una serie de modelos llamados LLaMA, \cite{touvron2023llama} que van desde 7B hasta 65B de parámetros, con un rendimiento competitivo en comparación con los mejores LLMs existentes.

Por ejemplo, LLaMA-13B supera a GPT-3 en la mayoría de las pruebas, a pesar de ser 10 veces más pequeño. Se espera que estos modelos democratizen el acceso y el estudio de los LLMs, ya que pueden ejecutarse en una sola GPU. Además, se asegura la compatibilidad con la fuente abierta al utilizar solo datos públicamente disponibles, a diferencia de otros modelos que dependen de datos no disponibles públicamente. El trabajo detalla las modificaciones realizadas en la arquitectura del $transformers$ \ref{sec:Transformers} y el método de entrenamiento, además de presentar el rendimiento de los modelos en comparación con otros LLMs en una serie de pruebas estándar. También se examinan los sesgos y la toxicidad codificados en los modelos, utilizando benchmarks de la comunidad de inteligencia artificial responsable.

El conjunto de datos de entrenamiento es una mezcla de varias fuentes, que cubren un conjunto diverso de dominios. Mayormente reutiliza fuentes de datos que se han utilizado para entrenar otros LLMs, con la restricción de utilizar solo datos públicamente disponibles y compatibles con la distribución abierta.

La tokenización de los datos se lleva a cabo con el algoritmo de codificación de bytes (BPE), utilizando la implementación de  $SentencePiece$ \footnote{El algoritmo $SentencePiece$ \cite{kudo2018sentencepiece} utiliza un enfoque basado en subpalabras, donde construye un vocabulario de subpalabras que se adaptan a la frecuencia de aparición en el corpus de entrenamiento. } . Dividimos todos los números en dígitos individuales y recurrimos a bytes para descomponer caracteres UTF-8 desconocidos. El tamaño total de nuestro conjunto de datos de entrenamiento contiene aproximadamente 1.4T de tokens después de la tokenización. La mayoría de los datos de entrenamiento se utilizan solo una vez durante el entrenamiento, con la excepción de los dominios de Wikipedia y Libros, sobre los cuales se realizan aproximadamente en dos épocas.
\subsection{Modelos LLM de Google AI}
A lo largo de los años Google AI ha desarrollado varios modelos LLM, todos ellos basados en Transformers, aunque con diferentes alcances y capacidades. A continuación se presentan estos modelos, aunque se explicaran con más profundidad a lo largo de la memoria. 
\subsubsection{LaMDA}
\label{sec:LaMDA}
LaMDA (Modelo de Lenguaje para Aplicaciones de Diálogo) fue el primer modelo LLM de Google AI. Su primera iteración fue anunciada durante la conferencia Google I/O de 2021, donde se presentó como un modelo conversacional de lenguaje. La segunda versión, presentada al año siguiente, introdujo mejoras y nuevas capacidades, como la generación de conversaciones originales sobre temas no previamente enseñados.

La atención sobre LaMDA aumentó cuando el ingeniero de Google, Blake Lemoine, afirmó que el chatbot se había vuelto sensible, lo que generó debates sobre la efectividad de la prueba de Turing para evaluar la inteligencia artificial general.

A diferencia de la mayoría de los modelos de lenguaje, LaMDA fue entrenado específicamente en diálogo. Durante su entrenamiento, adquirió sutilezas que distinguen las conversaciones abiertas de otras formas de lenguaje, como el sentido común.

La arquitectura Transformer utilizada por LaMDA es un modelo de solo decodificador, pre-entrenado en un corpus que incluye documentos y diálogos. Ha sido ajustado y probado con diferentes configuraciones de hiperparámetros, demostrando superar las respuestas humanas en ciertas áreas específicas.

\subsubsection{Bard}

Bard, anunciado por Google AI en 2022, se presenta como la evolución natural de su predecesor, LaMDA. Si bien este último ya destacaba por su capacidad para generar texto, traducir idiomas, escribir contenido creativo y responder a preguntas de forma informativa, Bard va un paso más allá. 

Bard se nutre de un conjunto de datos de texto y código aún más extenso que LaMDA, lo que le permite acceder y procesar información del mundo real a través de la Búsqueda de Google. Esto se traduce en respuestas más completas, precisas y actualizadas, ya que Bard se mantiene al día con los últimos acontecimientos y datos disponibles en la web. Su potencial creativo se expande a la escritura de diferentes tipos de contenido, desde poemas y código hasta guiones, piezas musicales, correos electrónicos y cartas. 

Además de la arquitectura Transformer, Bard incorpora otras técnicas de vanguardia como la atención y la decodificación autoregresiva. La atención permite a Bard centrarse en las partes más relevantes de un texto, mientras que la decodificación autoregresiva le permite generar texto palabra por palabra de forma coherente y fluida.

\subsubsection{Gemini}
Gemini es la última generación de modelos de lenguaje grandes de Google AI. Se anunció en 2024 y se basa en la arquitectura de Bard. Gemini es capaz de realizar todas las tareas que Bard puede hacer, y además tiene algunas características nuevas, como la capacidad de generar diferentes formatos de texto creativo, como poemas, código, guiones, piezas musicales, correo electrónico, cartas, etc. 

Su razonamiento multimodal avanzado le permite comprender y responder a preguntas complejas que involucran diferentes tipos de información, desde datos textuales hasta imágenes y vídeos. Gemini mejora constantemente sus habilidades y rendimiento a medida que se expone a nuevos datos y experiencias.

\section{Alucinaciones}
\label{sec:Alucinaciones}
En el campo de la inteligencia artificial (IA), se utiliza el término ``alucinación'' o ``alucinación artificiaL'' para describir respuestas de IA que no parecen estar justificadas por los datos de entrenamiento \cite{edwards2023chatgpt}. Este fenómeno, también conocido como confabulación o delirio, se refiere a la generación de respuestas o creencias que carecen de base sólida \cite{ji2022survey}. Por ejemplo, un chatbot alucinado podría ofrecer información falsa, como afirmar que los ingresos de Tesla fueron de 13.600 millones de dólares, un número aparentemente inventado \cite{lin2022trick}. Otros ejemplos de alucinaciones serían la generación respuestas inventadas como se puede ver en la imagen \ref{img:alucinacion}, donde ChatGPT genera un resumen de un artículo que no existe, o la adición de elementos en el estudio de imágenes, como se ve en la figura\ref{img:alucinacion2}.

\begin{figure}[h]
	\centering
	\includegraphics[scale=1]{Imagenes/ejemploAlucinacionGPT}
	\caption{Ejemplo de alucinación de ChatGPT \cite{wikialucinacion}}
	\label{img:alucinacion}
\end{figure}

\begin{figure}[h]
	\centering
	\includegraphics[scale=0.6]{Imagenes/alucinacionImagenes}
	\caption{Ejemplo de alucinación en una imagen \cite{rohrbach2023object}}
	\label{img:alucinacion2}
\end{figure}
Es importante destacar que, aunque se usa el término ``alucinación'' por analogía con la psicología humana, la diferencia fundamental radica en que las alucinaciones de IA se relacionan con respuestas injustificadas más que con percepciones falsas. Algunos investigadores señalan que este término antropomorfiza de manera poco razonable a los ordenadores.

El problema de las alucinaciones de IA se volvió prominente alrededor de 2022 con la introducción de modelos grandes de lenguaje como ChatGPT \citep{zhuo2023exploring}. Los usuarios expresaron preocupación por la capacidad de estos bots para insertar falsedades aparentemente plausibles en sus respuestas. En 2023, los analistas reconocieron las alucinaciones frecuentes como un desafío significativo en la tecnología de modelos de lenguaje \citep{leswing2023microsoft}.

Se cuestiona la confiabilidad del contenido generado por inteligencia artificial en el ámbito científico \citep{machinmastromatteo2023implicaciones}. Según Spinak (2023), los modelos de lenguaje de IA pueden percibir patrones que son imperceptibles para los humanos, lo que resulta en resultados inesperados o incorrectos, fenómeno conocido como ``alucinación'' \citep{spinak2023alucinaciones}. Estas alucinaciones pueden llevar a la producción de contenido falso, especialmente fuera de sus dominios específicos o al tratar con temas complejos o ambiguos, lo cual puede ser lingüísticamente plausible pero no científicamente preciso \citep{sage2023chatgpt}.

\section{Estado actual del procesamiento del lenguaje}
En los últimos años, el avance del procesamiento del lenguaje ha sido notable, con numerosas ventajas, aunque también con riesgos asociados al desarrollo de software.

Por un lado, los modelos de inteligencia artificial evolucionan rápidamente, generando frecuentes actualizaciones etiquetadas como ``última versión estable''. Esta denominación se refiere a la versión más reciente del modelo que ha sido probada y validada para su uso generalizado, ofreciendo calidad y fiabilidad.

Por otro lado, la IAs se enfrentan al problema de la regulación. La presidenta de la Comisión Europea, Ursula von der Leyen, ha subrayado que la IA está transformando nuestras vidas y ha enfatizado la importancia de un enfoque sensato y generalizado para beneficiar a la economía y la sociedad \citep{ComisionEuropea-ComunicadoPrensa-LeyIA}. La recién aprobada Ley de IA de la UE, considerada como el primer marco global en esta materia, busca promover la innovación responsable al regular los riesgos identificados y garantizar la seguridad y los derechos fundamentales de las personas y las empresas.

Este enfoque se basa en evaluar el riesgo de los sistemas de IA: aquellos de bajo riesgo, como los sistemas de recomendación, tendrán libertad y ninguna obligación, mientras que los de alto riesgo, como infraestructuras críticas, sistemas de salud y aplicaciones policiales, deberán cumplir requisitos estrictos de mitigación, calidad de datos y supervisión humana. Además, se prohibirán los sistemas de IA que representen una amenaza clara para los derechos fundamentales, como la manipulación del comportamiento humano. Se introducirán normas específicas para garantizar la transparencia en los modelos de IA de uso general, junto con multas para las empresas que no cumplan con las regulaciones.

En diciembre de 2023, se elaboró un borrador sobre la regulación de la IA por parte del Consejo de la Unión Europea y el Parlamento Europeo. Finalmente, el 2 de febrero de 2024, se aprobó la primera ley del mundo sobre inteligencia artificial: la Ley de IA. Después de una reunión en Bruselas, los embajadores de los 27 Estados miembros dieron su visto bueno político a esta normativa, tras la presentación en enero de la versión final del texto y la creación de la Oficina Europea de Inteligencia Artificial. Aunque algunos países mostraron oposición hasta el último momento, el camino continuó su curso y se prevé que entre 2024 y 2030 todos los países adopten la Ley de IA \citep{ElDerecho-LeyIA}.

Los constantes cambios en los modelos y las modificaciones en las legislaciones nacionales suponen un desafío para los desarrolladores de software. Durante el desarrollo de este proyecto, nos enfrentamos a diversas situaciones derivadas de estos sucesos. Por un lado, tuvimos que buscar una alternativa a la API de Bard, que se transformó en Gemini entre diciembre de 2023 y febrero de 2024, y por otro lado, hacer frente a las limitaciones impuestas por las diferentes leyes, que se solvento mediante el uso de VPN.

De hecho, la situación cambia tan rápidamente, que en la recta final del proyecto hemos observado varios cambios en los términos y condiciones de uso de la API de Gemini a nivel mundial. De hecho, la próxima actualización de los términos de uso será el 22 de mayo de 2024.

En conclusión, en el campo del desarrollo del lenguaje y la extracción de información a partir de imágenes, la IA está experimentando numerosos avances en los últimos años. Sin embargo, es fundamental tener en cuenta que el uso de los distintos modelos está sujeto a cambios tanto en las versiones y sus características, como en las leyes y términos y condiciones de uso.

\section{Otros trabajos relacionados}
\subsection{Proyecto Cantor}
El proyecto CANTOR (Composición automática de narrativas personales como apoyo a terapia ocupacional basada en reminiscencia) en el que se enmarca este trabajo, desarrolla herramientas digitales utilizando tecnologías de Inteligencia Artificial para construir automáticamente historias de vida que puedan ser reexaminadas posteriormente como apoyo a las terapias ocupacionales de pacientes con demencias.

CANTOR está financiado por el Ministerio de Ciencia e Innovación, en colaboración entre académicos de la Universidad Complutense de Madrid y la Universidad de La Coruña. El objetivo de CANTOR es desarrollar herramientas que faciliten la terapia ocupacional basada en reminiscencia para mejorar la calidad de vida de pacientes con deterioro cognitivo.

En este ámbito se han elaborado varios Trabajos de Fin de Grado en la Facultad de Informática de la UCM. Paso a referir algunos de ellos relacionados con este trabajo. 

\subsubsection{Generación de historias de vida usando técnias de Deep Learning}
\label{sec:trabajocristina}
En el curso 2021-2022, la compañera María Cristina Alameda Salas \citep{cristinaalameda}, en su trabajo de fin de grado, Generación de historias de vida
usando técnicas de Deep Learning, desarrolló un sistema basados en técnicas
de Deep Learning que de soporte a la generación de historias de vida. Partiendo de unos datos de entrada en forma de datos estructurados de tipo biográfico, ese trabajo permite la construcción de un sistema de generación de lenguaje natural, transformador de los datos de entrada a un escrito fluido y coherente, que abarque la representación de los datos de partida de manera completa, sin incorrecciones y lo más cercana posible a una redacción humana. Nuestro objetivo ahora sería el desarrollo de un programa que interactuara con el usuario y nos permitiera obtener toda esa información biográfica que da lugar a las historias de vida. 
\subsubsection{Extracción de preguntas a partir de imágenes para personas con problemas de memoria mediante técnicas de Deep Learning}

En 2021, en la UCM se desarrollo un trabajo de extracción de preguntas a partir de imagenes con técnicas de Deep Learning \citep{boto2021extraccion}. Este proyecto ayuda a las personas con problemas de memoria a recordar aspectos de su vida utilizando técnicas de IA como redes convolucionales y recurrentes. Para lograrlo se desarrolló un sistema capaz de extraer preguntas de fotografías que puedan representar recuerdos para las personas con problemas de memoria utilizando un bot que simula una sesión de terapia de reminiscencia.
El usuario ha de enviar fotografías al bot y este se encargará de enviarle, una a una, las preguntas generadas por la red neuronal. En este momento, el usuario deberá recordar todo lo posible sobre la imagen para poder responder a las preguntas y conseguir ejercitar su memoria.

\subsubsection{Extracción de información personal a partir de redes sociales para la creación de un libro de vida}
Este proyecto, desarrollado por \cite{aguilera2021extraccion}, tiene como objetivo principal ayudar a terapeutas ocupacionales en el tratamiento de pacientes con problemas de memoria, especialmente aquellos relacionados con el deterioro cognitivo asociado con la edad. Se propone la creación de una herramienta para desarrollar un libro de vida.

La herramienta combina técnicas de extracción y tratamiento de datos de diferentes redes sociales proporcionadas por el paciente, almacenándolos en una base de datos SQL para obtener la información más relevante. Esta información se utilizará para crear el libro de vida, que se presentará en una interfaz web desarrollada con React. La interfaz permitirá visualizar fácilmente los datos recopilados, utilizando tablas, mapas, líneas de tiempo y galerías de fotos.

\subsubsection{Generación de historias a partir de una base de conocimiento}
En este proyecto, desarrollado por \cite{lucia_latorre_magaz}, se construyó una aplicación para crear relaciones entre palabras e imágenes,
partiendo de unas palabras determinadas. Con las relaciones establecidas la aplicación genera estadísticas a través de las cuales puede evaluarse el progreso del paciente. En cada sesión se trata un tema concreto, pudiéndose elegir el tipo de sesión entre Sesión palabras (se elige una categoría de palabras), Sesión Progreso (se visualiza el avance del paciente a través de estadísticas agrupadas por categorías) o Sesión imágenes (donde se asocia una imagen a un concepto y una categoría).
\subsubsection{Recuerdame 1.0}
Recuerdame 1.0 \citep{recuerdame1.0} presenta la creación de una aplicación que facilite a los terapeutas la realización de terapias basadas en reminiscencia para tratar a pacientes con alzheimer, haciéndolas más ágiles y rápidas.

La aplicación creada es una aplicación web responsive, con una estructura Modelo Vista Controlador creada mediante lenguajes como HTML, CSS, PHP, JavaScript. La aplicación tiene una usabilidad aceptable, pero tiene detalles que mejorar y algunas funcionalidades que no pudieron ser desarrolladas por la falta de tiempo. 

\subsubsection{Recuerdame 2.0}
Durante el curso 2022-2023, la aplicación recuerdame 1.0 fue mejorada dando lugar a recuerdame 2.0 \citep{recuerdame2.0}, para ofrecer una experiencia terapéutica más enriquecedora a pacientes con problemas de memoria. Las mejoras incluyeron la optimización basada en la retroalimentación de usuarios finales, la narración mejorada de Historias de Vida, la generación automática de resúmenes, la integración de terapia con un bot, la mensajería entre terapeutas y cuidadores, y la capacidad de generar vídeos de Historias de Vida. Además, se realizó una evaluación exhaustiva del sistema en instituciones médicas y residencias de ancianos para validar su eficacia.

\subsection{Celia}

Celia, \footnote{\href{https://www.ambito.com/tecnologia/asi-es-celia-la-inteligencia-artificial-adultos-mayores-que-puede-detectar-indicios-alzheimer-n5921639}{Así es Celia, la inteligencia artificial para adultos mayores que puede detectar indicios de alzheimer}} es un chatbot impulsado por inteligencia artificial (IA) desarrollado por la compañía Atlantic, con el respaldo de la Xunta de Galicia en España. Este chatbot tiene como objetivo acompañar, entretener y brindar asistencia a las personas mayores y dependientes, y se destaca por su capacidad para detectar indicios y patrones de enfermedades neurodegenerativas, como el Alzheimer, mediante el análisis de la voz del usuario.

A diferencia de otros asistentes de conversación como Alexa o Siri, Celia va más allá al utilizar herramientas biométricas para medir y monitorear parámetros indicativos no solo de enfermedades neurológicas, sino también de condiciones emocionales como la ansiedad y la depresión.

Celia está disponible para su uso a través de tres plataformas: WhatsApp, la versión web y una aplicación oficial disponible actualmente solo para dispositivos Android. Los usuarios pueden interactuar con Celia a través de mensajes de texto o de voz, y la instalación es sencilla, lo que permite un acceso rápido y eficiente a este recurso tecnológico.

Una característica destacada de Celia es su capacidad para tomar la iniciativa en las conversaciones y proponer actividades sin necesidad de instrucciones. Además, ofrece la posibilidad de establecer recordatorios para citas médicas o la toma de medicamentos, brindando un apoyo integral en la gestión de la salud de los usuarios.

\chapter{Tecnologías utilizadas}
\label{cap:TecnologiasUtilizadas}

El capítulo actual se enfoca en detallar las tecnologías empleadas en la construcción y el despliegue del chatbot diseñado para asistir en terapias de reminiscencia. También se analizarán las herramientas y metodologías utilizadas en el proceso de desarrollo de prototipos, que se presenta en el capítulo \ref{cap:Desarrollo de prototipos}, así como su integración con los conceptos y conocimientos presentados en el estado de la cuestión.

En cada sección del capítulo, se llevará a cabo un análisis de las diferentes alternativas consideradas para la construcción de cada uno de los módulos que componen el chatbot. Desde la evaluación de diversas API's y bibliotecas de Procesamiento del Lenguaje Natural (PLN) hasta la exploración de las múltiples opciones disponibles para el desarrollo de la interfaz de usuario.

Esta metodología permitirá proporcionar un contexto completo y comprensible que facilitará la comprensión de las decisiones tomadas a lo largo del desarrollo del proyecto, como se detalla en el capítulo \ref{cap:Desarrollo de prototipos}. 
\newpage


\section{Bibliotecas de Procesamiento del Lenguaje en Python}

\subsection{NLTK}
La biblioteca NLTK (Natural Language Toolkit) \footnote{\href{https://www.nltk.org/}{Página oficial de NLTK}} ofrece una amplia gama de herramientas y recursos para tareas de PLN.

En primer lugar, NLTK permite realizar tareas como la tokenización y el  etiquetado POS (Part-Of-Speech tagging). Al utilizar las herramientas de tokenización, podemos dividir el texto en unidades más pequeñas, lo que facilita el análisis y la comprensión. El etiquetado POS asigna etiquetas gramaticales a cada palabra en el texto, lo que nos permite identificar la función de cada palabra en la oración. \\

\begin{lstlisting}[style=SpyderStyle, caption={Ejemplo de código en Python. Se pueden consultar más ejemplos en \cite{bird2009natural}}, captionpos=b, label={lst:python},breaklines = true]
	sentence = "Reminiscence Therapy involves the discussion of past activities using prompts like photos."
	tokens = nltk.word_tokenize(sentence)
	tagged = nltk.pos_tag(tokens)
	tagged[0:len(tagged)]
\end{lstlisting}

En este código $nltk$ tokeniza la oración introducida y etiqueta cada $token$ indicando la categoría sintáctica de cada $token$ como sigue:

\begin{itemize}
	\item NNP: Nombre propio singular
	\item NN: Nombre, singular o sustantivo singular
	\item VBZ: Verbo tercera persona del singular presente
	\item DT: Determinante
	\item IN: Preposición o oración subordinada
	\item JJ: Adjetivo
	\item NNS: Nombre plural 
	\item VBG: Verbo, gerundio o participio
	\item . : Signo de puntuación
\end{itemize}

\begin{lstlisting}[style=SpyderStyle, caption={Tokenización y etiquetado con nltk}, captionpos=b, label={lst:python},breaklines = true]
	>>>[('Reminiscence', 'NNP'), ('Therapy','NNP'), ('involves','VBZ'), ('the', 'DT'), ('discussion', 'NN'), ('of', 'IN'), ('past', 'JJ'), ('activities', 'NNS'), ('using', 'VBG'), ('prompts', 'NNS'), ('like', 'IN'), ('photos', 'NNS'), ('.', '.')]
	
\end{lstlisting}

Otras de las funcionalidades que nos permite esta biblioteca es el análisis sintáctico o lematización. Por ejemplo, nos permite la obtención de árboles sintácticos, lo que permite visualizar la estructura gramatical de las oraciones, facilita el análisis y la interpretación del texto. \\

\begin{figure}[h]
	\centering
	\includegraphics[width=0.9\textwidth]{Imagenes/arbolsintactico}
	\caption{Árbol sintáctico generado con nltk}
	\label{fig:2}
\end{figure}

\begin{lstlisting}[style=SpyderStyle, caption={Análisis sintáctico y lematización con nltk}, captionpos=b, label={lst:python},breaklines = true]
	entities = nltk.chunk.ne_chunk(tagged)
	nltk.download('treebank')
	from nltk.corpus import treebank
	t = treebank.parsed_sents('wsj_0001.mrg')[0]
	t
\end{lstlisting}


Además de estas características fundamentales, NLTK ofrece una serie de otras funcionalidades que amplían aún más su utilidad. Por ejemplo, incluye herramientas para la extracción de entidades nombradas, el análisis de sentimientos, la generación de texto y la traducción automática. Estas capacidades adicionales permiten abordar una amplia variedad de tareas en el procesamiento del lenguaje natural, desde la clasificación de texto hasta la generación de resúmenes automáticos y la traducción de idiomas. En resumen, NLTK es una herramienta invaluable para investigadores, estudiantes y profesionales que trabajan en el campo del PLN, ofreciendo una amplia gama de funcionalidades que facilitan el análisis, la comprensión y la manipulación del lenguaje humano.


\subsection{SpaCy}

% https://spacy.io/

SpaCy ofrece soporte para más de 25 idiomas y cuenta con 84 pipelines de entrenamiento. Utiliza el aprendizaje multi-tarea con modelos preentrenados como BERT, lo que permite un rendimiento avanzado en tareas de procesamiento del lenguaje natural. Sus componentes incluyen herramientas para el reconocimiento de entidades nombradas, etiquetado de partes del discurso, análisis de dependencias, segmentación de oraciones, clasificación de texto, lematización, análisis morfológico, vinculación de entidades y más.

\begin{lstlisting}[style=SpyderStyle, caption={Ejemplo de tokenización usando spaCy}, captionpos=b, label={lst:python},breaklines = true]
	import spaCy
	
	# Load English tokenizer, tagger, parser and NER
	nlp = spacy.load("en_core_web_sm")
	
	# Process whole documents
	text = ("Reminiscence Therapy involves the discussion of past activities using prompts like photos.")
	doc = nlp(text)
	
	# Analyze syntax
	print("Noun phrases:", [chunk.text for chunk in doc.noun_chunks])
	print("Verbs:", [token.lemma_ for token in doc if token.pos_ == "VERB"])
\end{lstlisting}

Este código carga el modelo preentrenado $"en\_core\_web\_sm"$ de spaCy, que incluye herramientas para tokenizar, etiquetar, analizar la sintaxis y reconocer entidades nombradas en textos en inglés. Luego, procesa el texto proporcionado y muestra las frases nominales identificadas utilizando la función $noun_chunks$ y los verbos lematizados utilizando la propiedad $lemma\_$. Este análisis gramatical permite identificar las partes clave del texto, como los sustantivos y las acciones descritas.

\begin{lstlisting}[style=SpyderStyle, caption={Resultado de tokenización usando spaCy}, captionpos=b, label={lst:python},breaklines = true]
	>>> Noun phrases: ['Reminiscence Therapy', 'the discussion', 'past activities', 'prompts', 'photos']
	Verbs: ['involve', 'use']
\end{lstlisting}

Además, es fácilmente ampliable con componentes y atributos personalizados, y es compatible con modelos personalizados en PyTorch, TensorFlow y otros frameworks. Spacy ofrece visualizadores integrados para la sintaxis y el reconocimiento de entidades nombradas, y facilita el empaquetado, despliegue y gestión de flujos de trabajo de modelos. Con su precisión rigurosamente evaluada y su robustez, Spacy es una herramienta poderosa y versátil para el procesamiento del lenguaje natural.

La biblioteca spaCy cuenta con diferentes componentes que interactúan entre sí esuchando la salida unos de otros para mejorar su procesamiento ($listener$). Además, existen una serie de reglas y dependencias entre los componentes. Por ejemplo, el módulo $atribute\_ruler$ indica proporciona reglas de etiquetado a $tagger$.

\begin{figure}[h]
	\centering
	\includegraphics[width=0.9\textwidth]{Imagenes/spaCy}
	\caption{Árbol sintáctico generado con nltk}
	\label{fig:1}
\end{figure}

El gráfico representa la estructura de una tubería de procesamiento de lenguaje natural (NLP) en spaCy, mostrando la secuencia de componentes y sus interacciones.

\begin{itemize}
	\item tok2vec: Este componente convierte los tokens en vectores de palabras, que capturan el significado semántico de las palabras en el contexto de la oración.
	\item tagger: El $tagger$ asigna etiquetas gramaticales a cada token en el texto, como partes del discurso (POS).
	\item  parser: El analizador sintáctico analiza la estructura sintáctica del texto, identificando las relaciones de dependencia entre las palabras. 
	\item $attribute\_ruler$: Este componente aplica reglas para agregar atributos adicionales a los tokens, como excepciones de lema y POS, y manejar espacios en blanco de manera coherente.
	\item $lemmatizer$: El lematizador determina la forma base de cada palabra (su lema) en función de su contexto y su parte del discurso.
	\item $ner/tok2vec$: El componente de reconocimiento de entidades (NER) identifica entidades nombradas en el texto, como nombres de personas, lugares o organizaciones. En algunos modelos, este componente comparte la representación de vectores de palabras (tok2vec) con otros componentes para mejorar la coherencia y la precisión de las predicciones.
\end{itemize}
En resumen, este gráfico muestra cómo los componentes de spaCy interactúan entre sí para realizar tareas de procesamiento de lenguaje natural, aprovechando la información compartida y las reglas definidas para mejorar la precisión y la coherencia del análisis lingüístico.



\section{APIs de procesamiento del lenguaje}

Las APIs de procesamiento del lenguaje son conjuntos de herramientas y servicios que integran múltiples funcionalidades relacionadas con el PLN en sus aplicaciones y sistemas. Es decir, son herramientas que aúnan y ofrecen funcionalidades como el análisis de sentimientos, el reconocimiento de entidades o la tokenización. Frente a las bibliotecas y modelos presentados anteriormente presentan la ventaja de que se pueden usar sin necesidad de desarrollar desde cero algoritmos o modelos, lo que facilita su uso. Estas características hacen que este tipo de APIs sean comúnmente usadas en variedad de aplicaciones, desde chatbots hasta sistemas de recomendación. 

\subsection{Bard}
Bard es una API de procesamiento del lenguaje natural desarrollada por Google con el objetivo de ofrecer respuestas conversacionales coherentes y relevantes a través de interacciones de mensajes. Basada en LaMDA, un modelo de lenguaje experimental de Google, Bard compite directamente con ChatGPT en el campo del procesamiento del lenguaje natural, permitiendo realizar consultas y recibir respuestas sin necesidad de navegar por diferentes páginas web.

Google ha priorizado la accesibilidad y la transparencia en el desarrollo de Bard, ofreciendo modelos y recursos de PLN de código abierto que pueden ser utilizados y modificados por la comunidad. Inicialmente lanzado con un modelo reducido de LaMDA \ref{sec:LaMDA}, Bard busca ampliar su alcance y obtener comentarios para su mejora continua.
\begin{figure}[h]
	\centering
	\begin{verbatim}
prompt = "A partir del texto a continuación, que contiene información
 sobre una persona y damelo en una lista info donde
 info[nombre]:valor_atributo."

texto = "Hola mi nombre es Marta, tengo 22 años y soy de Zaragoza"
>> **Info[nombre:Marta;edad:22;ciudad:Zaragoza]** 
¿Hay algo más que pueda hacer por ti?
	\end{verbatim}
	\caption{ Ejemplo de uso de BARD.}
	\label{fig:ejemploBARD}
\end{figure}




Cómo se puede ver en el ejemplo, Bard es capaz de analizar la información proporcionada y generar respuestas coherentes y formateadas según las especificaciones dadas. Gracias a su capacidad para comprender el contexto y generar texto de manera precisa, Bard es una herramienta valiosa para tareas que requieren PLN, como la generación de respuestas conversacionales. Todo ello lo convierte en una opción ideal para una amplia gama de aplicaciones, desde chatbots hasta sistemas de asistencia virtual. 

Sin embargo, Bard ya no está disponible. En diciembre de 2023, Google fortaleció la capacidad de Bard al incorporar Gemini Pro en inglés, brindando habilidades más avanzadas de comprensión, razonamiento, resumen y codificación. Más adelante, en febrero de 2024, se anunció la expansión de Gemini Pro a más de 40 idiomas y se oficializó el cambio de nombre de Bard a Gemini, lo que implicó descartar el primer modelo del proyecto desarrollado en Bard debido a la indisponibilidad de Gemini en España en ese momento.

\subsection{Gemma}

Gemma es una API de PLN desarrollada por OpenAI. Utiliza modelos de lenguaje basados en la arquitectura GPT (Generative Pre-trained Transformer) para una variedad de tareas de PLN, como generación de texto, análisis de sentimientos, clasificación de texto, y más. Gemma se destaca por su capacidad para generar texto coherente y de alta calidad en una variedad de estilos y tonos, así como por su facilidad de uso y su API intuitiva.

Una de las principales ventajas de Gemma es su rendimiento en tareas de generación de texto, donde ha establecido nuevos estándares de calidad y coherencia en muchos casos. Además, Gemma ofrece modelos pre-entrenados en varios dominios y lenguas, lo que facilita su integración en una variedad de aplicaciones de PLN. Sin embargo, debido a su enfoque en modelos de última generación, Gemma puede requerir recursos computacionales significativos y puede ser más difícil de entender y utilizar para usuarios principiantes en PLN.


En primer lugar, estos modelos se trabajaron en Google Collaborate aumentando el número de GPUs. De esta forma gemma tiene un buen comportamiento y genera respuestas adecuadas y coherentes. En concreto, una respuesta generada por \textit{gemma-7b} sería la siguiente.
\begin{figure}[h]
	\centering
	\begin{verbatim}
prompt =gemma_lm.generate("What is the meaning of life?",
 max_lenght = 64)
 >>> The question is one of the most important questions in the world.
  It's the question that has been asked by philosophers, theologians and
  scientist for centuries. And it's the question that has been asked by
  people who are looking for answer to their own lives. 
	\end{verbatim}
	\caption{ Ejemplo de uso de GEMMA.}
	\label{fig:ejemploBARD}
\end{figure}
Sin embargo, las limitaciones propias de Google Collaborate no permitían en la versión gratuita aumentar el número de GPUs de forma frecuente y en consecuencia tuve que estudiar el modelo en otro entorno. Para ejecutarlo de forma local y obtener un buen comportamiento es necesario instalar Linux y descargar el modelo de Hugging face. 

Aunque \textit{gemma-2b} en la versión local instalada desde Hugging Face en Linux tiene un buen comportamiento, genera respuestas incoherentes que hacen de este modelo poco útil para nuestro proyecto. La versión \textit{gemma-7b} genera respuestas mucho mejores pero tiene la enorme desventaja de que ocupa una gran cantidad de espacio en memoria.

\subsection{GPT API}

GPT API es una API de PLN desarrollada por OpenAI. Utiliza modelos de lenguaje basados en la arquitectura GPT (Generative Pre-trained Transformer) para una variedad de tareas de PLN, como generación de texto, análisis de sentimientos, clasificación de texto, y más. GPT API se destaca por su capacidad para generar texto coherente y de alta calidad en una variedad de estilos y tonos, así como por su facilidad de uso y su API intuitiva.

Una de las principales ventajas de GPT API es su rendimiento en tareas de generación de texto, donde ha establecido nuevos estándares de calidad y coherencia en muchos casos. Además, GPT API ofrece modelos pre-entrenados en varios dominios y lenguas, lo que facilita su integración en una variedad de aplicaciones de PLN. Sin embargo, debido a su enfoque en modelos de última generación, GPT API puede requerir recursos computacionales significativos y puede ser más difícil de entender y utilizar para usuarios principiantes en PLN.

Sin embargo para obtener el comportamiento que se necesitaba en este trabajo debía ser entrenada, y debido a las limitaciones hardware esto suponía una cantidad de tiempo inviable. 

\subsection{Rasa}
Entre las opciones que se barajaron para seguir desarrollando el proyecto se encuentra Rasa. Rasa es una plataforma de código abierto diseñada para el desarrollo de chatbots y asistentes virtuales conversacionales. Utilizando técnicas de procesamiento de lenguaje natural (NLP) y aprendizaje automático, Rasa permite a los desarrolladores crear sistemas de diálogo inteligentes y personalizados. Una de las principales ventajas de Rasa es su flexibilidad y personalización, ya que los desarrolladores tienen control total sobre el comportamiento y la lógica de sus chatbots. Además, Rasa proporciona herramientas robustas para la gestión del diálogo, la comprensión del lenguaje natural y la integración con otros sistemas. Sin embargo, una posible desventaja de Rasa es su curva de aprendizaje, ya que requiere un conocimiento sólido de NLP y aprendizaje automático para aprovechar al máximo su potencial. Además, debido a su naturaleza de código abierto, puede requerir más tiempo y recursos para implementar y mantener en comparación con otras soluciones comerciales. Sin embargo, aunque rasa no es la api más potente en cuánto a generación de texto, tiene numerosas aplicaciones que resultan interesantes. Por ejemplo, gracias a la api de rasa es fácil volcar la interfaz de código en Python sobre la interfaz de Telegram. 

\subsection{Gemini}

%De aqui quiero sacar varias páginas, ver el tutorial y ir metiendo todo

Gemini es una API de procesamiento del lenguaje natural (PLN) desarrollada por Google que permite a los usuarios interactuar con modelos de lenguaje avanzados para generar texto coherente y relevante en respuesta a consultas y solicitudes. Utiliza modelos de lenguaje de última generación entrenados por Google, que son capaces de comprender y generar texto en varios idiomas y contextos. Los usuarios pueden enviar texto de entrada a través de la API y recibir respuestas generadas por los modelos de Gemini. Ofrece varios modelos para satisfacer diferentes necesidades y casos de uso, entre los que se encuentran:
\begin{itemize}[label=$\bullet$, leftmargin=*]
	\item \textbf{gemini-pro}: Optimizado para entradas de texto.
	\item \textbf{gemini-pro-vision}: Optimizado para entradas multimodales de texto e imágenes.
\end{itemize}

Gemini puede utilizarse para una variedad de aplicaciones, incluyendo generación de texto a partir de entradas bien sean de texto, o imágenes, conversaciones de varios turnos (chats) o para la obtención de embeddings para modelos del lenguaje. 

Para configurarlo, en primer lugar hay que ejecutar el programa mibot.py estando en telegram en la conversaa conversación de telegram el comando /start para comenzar la conversación y ya. 

Sin embago, y pese a las grandes funcionalidades de todas estas alternativas nos hemos decantado por hacer una interfaz con telegram desarrollando nuestro propio chatbot usando rasa. 

Para desarrollar el chatbot de telegram me he decantado por usar la interfaz de telegram para la cuál se necesita la API de Rasa. Las principales ventajas que ofrece esta herramienta es la facilidad del manejo de la interfaz pues telegram es una herramienta muy conocida con la que los terapeutas pueden estar más familiarizados. Además, esto nos permite también usar la versión del chatbotyayo para móvil. 

Para crear está interfaz hay que: 
1. Instalar rasa
2. Instalar telegram 
2. Obtener una api de rasa
4. crear un nuevo chatbot desde telegram con @botFather
5. Enviar a tu chatbot el comando /start

Una vez seguidos todos estos pasos ya puedes comenzar a interactuar con la API de rasa. 


\section{Respuestas de Gemini}

El modelo más apropiado para el procesamiento de texto es $gemini-pro$. La estructura de las respuestas de este modelo es la siguiente. 

\begin{lstlisting}[style=SpyderStyle, caption={Estructura de una respuesta de Gemini}, captionpos=b, label={lst:python},breaklines = true]
	{
		"candidates": [
		{
			"content": {
				"parts": [
				{
					"text": string
				}
				]
			},
			"finishReason": enum (FinishReason),
			"safetyRatings": [
			{
				"category": enum (HarmCategory),
				"probability": enum (HarmProbability),
				"blocked": boolean
			}
			],
			"citationMetadata": {
				"citations": [
				{
					"startIndex": integer,
					"endIndex": integer,
					"uri": string,
					"title": string,
					"license": string,
					"publicationDate": {
						"year": integer,
						"month": integer,
						"day": integer
					}
				}
				]
			}
		}
		],
		"usageMetadata": {
			"promptTokenCount": integer,
			"candidatesTokenCount": integer,
			"totalTokenCount": integer
		}
	}
\end{lstlisting}

\begin{itemize}
	\item \textbf{text}	El texto generado.
	\item \textbf{finishReason}	El motivo por el que el modelo dejó de generar tokens. Si está vacío, el modelo no dejó de generar los tokens. El motivo puede ser cualquiera de los siguientes:
	\begin{enumerate}
		\item $FINISH\_REASON\_UNSPECIFIED$: no se especifica el motivo de finalización.
		\item $FINISH\_REASON\_STOP$: punto de detención natural del modelo o secuencia de detención proporcionada.
		\item $FINISH\_REASON\_MAX\_TOKENS$: se alcanzó la cantidad máxima de tokens especificada en la solicitud.
		\item $FINISH\_REASON\_SAFETY$: la generación del token se detuvo porque la respuesta se marcó por motivos de seguridad. Ten en cuenta que Candidate.content está vacío si los filtros de contenido bloquean el resultado.
		\item $FINISH\_REASON\_RECITATION$: la generación del token se detuvo porque la respuesta se marcó para citas no autorizadas.
		\item $FINISH\_REASON\_OTHER$: todos los demás motivos que detuvieron el token
	\end{enumerate}
	
	\item \textbf{category}	La categoría de seguridad para la que se configura un umbral. Los valores aceptables son los siguientes:
	Haz clic para expandir las categorías de seguridad
	\begin{enumerate}
		\item $HARM\_CATEGORY\_SEXUALLY\_EXPLICIT$
		\item $HARM\_CATEGORY\_HATE\_SPEECH$
		\item $HARM\_CATEGORY\_HARASSMENT$
		\item $HARM\_CATEGORY\_DANGEROUS\_CONTENT$
	\end{enumerate}
	\item \textbf{probability}	Los niveles de probabilidad de daños en el contenido.
	\begin{enumerate}
		\item $HARM\_PROBABILITY\_UNSPECIFIED$
		\item $NEGLIGIBLE$
		\item $LOW$
		\item $MEDIUM$
		\item $HIGH$
	\end{enumerate}
	\item \textbf{blocked}	Una marca boolean asociada con un atributo de seguridad que indica si la entrada o salida del modelo se bloqueó. Si blocked es true, el campo errors en la respuesta contiene uno o más códigos de error. Si blocked es false, la respuesta no incluye el campo errors.
	\item \textbf{startIndex}	Un número entero que especifica dónde comienza una cita en el contenido.
	\item \textbf{endIndex}	Un número entero que especifica dónde termina una cita en content.
	\item \textbf{url}	Es la URL de una fuente de cita. Los ejemplos de una fuente de URL pueden ser un sitio web de noticias o un repositorio de GitHub.
	\item \textbf{title}	Es el título de una fuente de cita. Los ejemplos de títulos de origen pueden ser los de un artículo de noticias o un libro.
	\item \textbf{license}	Es la licencia asociada con una cita.
	\item \textbf{publicationDate}	La fecha en que se publicó una cita. Sus formatos válidos son YYYY, YYYY-MM y YYYY-MM-DD.
	\item \textbf{promptTokenCount}	Cantidad de tokens en la solicitud.
	\item \textbf{candidatesTokenCount}	Cantidad de tokens en las respuestas.
	\item \textbf{totalTokenCount}	Cantidad de tokens en la solicitud y las respuestas.
\end{itemize} 
\section{Almacenamiento de la información}	
\subsection{JSON}
El JSON, acrónimo de JavaScript Object Notation, es un formato ligero para estructurar datos que se asemeja a los mapas en la programación. Está diseñado para representar datos de manera legible para las máquinas y fácilmente interpretable por los humanos. Se utiliza ampliamente en el intercambio de datos entre aplicaciones web y en el manejo de respuestas de API. El formato JSON consta de pares de clave-valor, donde las claves son únicas y los valores pueden ser cadenas, booleanos, números, objetos JSON o matrices JSON.

En Python, el formato de datos más cercano a JSON es el diccionario. El módulo `json` de Python permite la conversión entre diccionarios, cadenas JSON y archivos JSON. Para leer un archivo JSON en Python, se utiliza la función \textit{json.load()} para cargar los datos del archivo en un diccionario. Para leer una cadena JSON en Python, se utiliza la función  \textit{json.loads()} para analizar la cadena y convertirla en un diccionario.

Para escribir datos en un archivo JSON, se utiliza la función \textit{json.dump()} para escribir un diccionario en un archivo JSON. También se puede utilizar la función \textit{json.dumps()} para convertir un diccionario en una cadena JSON y luego escribir esa cadena en un archivo. Ambas funciones permiten especificar la indentación para formatear el archivo JSON de manera legible. El módulo \textit{json} proporciona una forma conveniente de manipular datos JSON en Python, facilitando el intercambio de datos entre aplicaciones y su almacenamiento en archivos.

\subsection{RDF}

%Las tripletas RDF (Resource Description Framework) son una estructura de datos fundamental en la web semántica para representar información en forma de sujetos, predicados y objetos. Cada tripleta consiste en un sujeto que es una entidad, un predicado que describe la relación entre el sujeto y el objeto, y un objeto que puede ser una entidad o un valor. Por ejemplo, en la tripleta "Gato - es_un - Animal", "Gato" es el sujeto, "es_un" es el predicado y "Animal" es el objeto.

Estas tripletas RDF se utilizan para almacenar información de manera estructurada y semántica, lo que permite una representación más rica y significativa de los datos en la web. A través de vocabularios y ontologías, como RDF Schema (RDFS) y Web Ontology Language (OWL), se establecen relaciones y significados precisos entre los términos utilizados en las tripletas RDF.

Las tripletas RDF son ampliamente utilizadas en la web semántica para diversas aplicaciones, como la descripción de recursos y metadatos en la web, la integración de datos de diferentes fuentes, la creación de motores de búsqueda más inteligentes y la construcción de sistemas de recomendación personalizados. Además, RDF proporciona un marco estándar y flexible para representar conocimiento y facilita la interoperabilidad entre diferentes sistemas y aplicaciones en la web.

	
El Framework de Descripción de Recursos (RDF, por sus siglas en inglés) es un lenguaje de propósito general orientado a la representación de información en la web. Su uso se centra en el desarrollo de la web semántica, y tiene como finalidad describir los recursos de la misma de una manera no orientada a la legibilidad por parte de un humano, sino a la computación de la información contenido por un ordenador.

La web semántica es un proyecto de futuro en el que la información web tiene un significado exactamente definido y puede ser procesado por ordenadores. Por lo tanto, los ordenadores pueden integrar y usar la información disponible en la web. Más información sobre la web semántica puede ser encontrada en \cite{iswc2007}.

RDF está considerado como un lenguaje de metadatos, o "datos sobre datos", ya que con los símbolos de RDF añadimos metainformación a los datos que realmente nos interesan para poder interpretarlos de una manera exacta, es decir, de aquella que el autor de los datos (o, al menos, del autor del marcado RDF sobre estos datos) quería que estos fuesen interpretados.

RDF define los recursos mediante descripciones de los mismos, como puede verse en el listado \ref{fig:ejemploRDF1}. Puede encontrarse más información sobre la manera de la tecnología RDF de describir estos recursos en \cite{champin2002rdf}.

\begin{figure}[h]
	\centering
	\begin{verbatim}
		<rdf:Description
		rdf:about="http://www.recshop.fake/cd/Empire Burlesque">
		<cd:artist>Bob Dylan</cd:artist>
		<cd:country>USA</cd:country>
		<cd:company>Columbia</cd:company>
		<cd:price>10.90</cd:price>
		<cd:year>1985</cd:year>
		</rdf:Description>
	\end{verbatim}
	\caption{ Ejemplo de un recurso RDF.}
	\label{fig:ejemploRDF1}
\end{figure}


RDF no define clases de datos específicas para las aplicaciones. En vez de esto, el estándar RDF dispone de RDF Schema. Estos documentos definen nuevas clases, y relaciones entre ellas (herencia, agregación) de una manera muy similar a aquella con la que se acostumbra en la programación orientada a objetos. En la Figura \ref{fig:ejemploRDF2} podemos ver un ejemplo de un RDF Schema.

\begin{figure}[t]
	\centering
\begin{verbatim}
	<?xml version="1.0"?>
	<rdf:RDF
	xmlns:rdf= "http://www.w3.org/1999/02/22-rdf-syntax-ns#"
	xmlns:rdfs="http://www.w3.org/2000/01/rdf-schema#"
	xml:base= "http://www.animals.fake/animals#">
	<rdf:Description rdf:ID="animal">
	<rdf:type
	rdf:resource="http://www.w3.org/2000/01/rdf-schema#Class"/>
	</rdf:Description>
	<rdf:Description rdf:ID="horse">
	<rdf:type
	rdf:resource="http://www.w3.org/2000/01/rdf-schema#Class"/>
	<rdfs:subClassOf rdf:resource="#animal"/>
	</rdf:Description>
	</rdf:RDF>
\end{verbatim}
	\caption{ Ejemplo de un recurso RDF SCHEMA.}
	\label{fig:ejemploRDF2}
\end{figure}

\section{Desarrollo de la interfaz}

PyQt, wxPython y Kivy son opciones populares para la implementación de interfaces gráficas, cada una con sus propias ventajas y desventajas.

PyQt es conocido por su completo conjunto de widgets, lo que te permite crear interfaces gráficas complejas y altamente personalizadas. Sin embargo, puede tener una curva de aprendizaje más pronunciada debido a su complejidad y sintaxis más compleja.

Por otro lado, wxPython ofrece una sintaxis más simple y fácil de entender, lo que puede ser beneficioso si estás empezando o prefieres un enfoque más directo. Aunque tiene menos widgets y funcionalidades avanzadas que PyQt, sigue siendo una opción sólida con una comunidad activa que proporciona soporte.


Kivy destaca por su diseño adaptable, diseñado para crear aplicaciones con interfaces gráficas que funcionan en una amplia gama de dispositivos. Utiliza un lenguaje de marcado declarativo que permite definir la interfaz de usuario de manera intuitiva y separada del código Python. Sin embargo, puede tener menos documentación y recursos disponibles en comparación con PyQt y wxPython.

\begin{figure}[h]
	\centering
	\includegraphics[width=0.4\textwidth]{Imagenes/PyQT5_Interfaz}
	\caption{Ejemplo de interfaz generada con PyQT5}
	\label{fig:interfazPYQT5}
\end{figure}

\section{Programación orientada a objetos}
La Programación Orientada a Objetos (POO) ha ganado una popularidad significativa en la comunidad de programación debido a su capacidad para desarrollar aplicaciones más robustas, flexibles y fáciles de mantener. Este paradigma se basa en la organización de programas como una colección de objetos interconectados, cada uno con su propio conjunto de datos y funcionalidades. En este artículo, exploraremos los conceptos clave de la POO, cómo implementarla en diversos lenguajes y cómo aprovechar sus ventajas para construir aplicaciones sólidas y flexibles.

La POO es un paradigma de programación que se basa en la idea de clases y objetos. Se utiliza para estructurar programas de software en piezas simples y reutilizables de código, donde una clase actúa como una plantilla para crear múltiples instancias de objetos. Este enfoque invita a considerar las entidades dentro del contexto del problema a resolver, como libros, bibliotecarios y usuarios, y representarlas como objetos con propiedades y comportamientos.

La POO se inspira en la forma en que percibimos y entendemos el mundo que nos rodea. Cada entidad se convierte en un objeto con sus propios atributos y métodos, y la interacción entre estos objetos es fundamental. La encapsulación, abstracción, herencia y polimorfismo son los principios fundamentales de la POO que permiten crear aplicaciones más organizadas, reutilizables y mantenibles.

La POO permite la reutilización del código, evita la duplicación, protege la información a través de la encapsulación y facilita el trabajo en equipo al minimizar la posibilidad de duplicar funciones. Además, proporciona una estructura más clara y modular para el desarrollo de software, lo que facilita el mantenimiento y la escalabilidad a medida que los requisitos evolucionan.

La Programación Orientada a Objetos es esencial en el diseño de aplicaciones y programas informáticos modernos. Ofrece numerosas ventajas, como la reutilización del código, la modularidad y la facilidad de mantenimiento, lo que la convierte en una opción ideal para resolver desafíos de programación complejos. Sin embargo, requiere una planificación cuidadosa y un análisis detallado de los requisitos para aprovechar al máximo sus beneficios.



 





\chapter{Desarrollo de prototipos}
\label{cap:Desarrollo de prototipos}
El presente capítulo tiene como objetivo contar, en orden temporal, como ha sido la construcción del chatbot de ayuda a la terapia de reminiscencia. Así, habla sobre los detalles, características y las tecnologías involucradas en cada uno de los prototipos que han llevado el proyecto hasta el prototipo final. 
%PRIMERA VERSIÓN EN BARD
%PRIMERA VERSIÓN EN GEMMA
%PRIMERA VERSIÓN GEMINI - GOOGLE COLLABORATE
%REFACTORIZACIÓN PARA HACER VERSIÓN LOCAL
%TELEGRAM
%GENERAR LAS ETAPAS 
%IMAGENES
%WHISPER
%GRÁFOS
%GENERACIÓN DE HISTORIAS DE VIDA

\section{Prototipo con preguntas predefinidas}
%Parsea la información utilizando BARD
\section{Prototipo con preguntas predefinidas usando Gemma}
\section{Prototipo con preguntas predefinidas usando}
\section{Gemini}
\section{Gemini preguntas}
\section{Gemini almacenamiento de información}

\chapter{ChatBot final}
\label{cap:ChatBot final}
\section{Desarrollo de las funciones}

\chapter{Conclusiones y Trabajo Futuro}
\label{cap:conclusiones}
En el presente capítulo se describen las conclusiones a las que se han llegado tras meses de trabajo en un proyecto de construcción de una herramienta conversacional de ayuda a la terapia de reminiscencia. Pese al desarrollo exponencial que esta sufriendo la IA en el campo del procesamiento del lenguaje, la provisionalidad de los módelos y los constantes cambios en las legislaciones de los territorios han llevado a diversos problemas y en consecuencia a un estudio mucho más amplio de la situación actual y de las diferentes alternativas. 

En este capítulo, no solo se pretende hablar de las conclusiones acerca de la herramienta desarrollada, si no a todas las conclusiones asociadas al estudio teórico previo. 

Por otro lado, se decribiran los posibles trabajos futuros que pueden considerarse como fruto de la investigación.


\section{Resultado final}
Como resultado del desarrollo de este proyecto, se ha creado una herramienta conversacional para apoyar la terapia de reminiscencia. La principal función de este bot es buscar y almacenar información relevante para generar historias de vida utilizando otras herramientas. La interfaz de usuario intuitiva y fácil de usar hace que esta herramienta sea accesible no solo para terapeutas, sino también para familiares y pacientes que estén familiarizados con dispositivos móviles y computadoras.

Aunque inicialmente diseñada como una herramienta terapéutica para recopilar información sobre la vida del paciente, esta herramienta tiene múltiples usos. Por ejemplo, las versiones portátiles, como la aplicación móvil o para tabletas, permiten su uso como entretenimiento durante viajes largos que pueden resultar desafiantes para los pacientes. Además, la facilidad de uso también permite que los pacientes la utilicen en momentos de soledad, manteniendo conversaciones interesantes sobre su vida que estimulan la mente y les proporcionan compañía virtual.

\section{Trabajo futuro}

Como punto final de este trabajo se exponen aquellas posibles líneas abiertas de desarrollo e investigación, que han surgido tras el análisis del proyecto, y que podrían ser estudiados con la finalidad de mejorar el proyecto. 

En primer lugar, aunque este trabajo genere una pequeña historia de vida como resultado final de la conversación sería interesante conectar este trabajo con el proyecto explicado en la sección \ref{sec:trabajocristina} ``Generación de historias de vida usando técnias de Deep Learning''. Para ello sería interesante transformar toda la información que se obtiene mediante la conversación con este chatbot en un conjunto de entrada con el formato adecuado para la aplicación desarrollada en ese trabajo. De esta forma, se generarían historias de vida más fieles a la vida real de la persona, y se completaría el ciclo completo de trabajo previo a la terapia de reminiscencia. 

Por otro lado, otra modificación que podría resultar interesante sería la adaptación del modelo a múltiples lenguajes de forma que se amplié el público objetivo de la aplicación. La API de Gemini esta disponible en diferentes lenguajes, pero el manejo de todas las preguntas predefinidas tendría que ser traducido para la interacción con el usuario. 

El desarrollo de la API de Gemini, y el acceso de nuevas versiones traerá consigo nuevas posibilidades. Entre ellas, destaca la integración de vídeos para enviar al chatbot y que se analicen con el objetivo de obtener más información bibliográfica. También puede ser interesante considerar la alternativa del reconocimiento de voz de forma que además de la interfaz de Telegram, se pudiera desarrollar una interfaz de voz donde el paciente pudiera hablar y escuchar como si se tratará de una llamada telefónica.

\begin{otherlanguage}{english}
	\chapter*{Conclusions and Future Work}
\label{chap:conclusions}
This chapter describes the conclusions reached after months of work on a project to build a conversational tool to support reminiscence therapy.

Despite the exponential development of AI in the field of language processing, the provisionality of models and constant changes in legislation across different regions have caused various issues, leading to a more extensive study of the current situation and different alternatives.

In this chapter, we aim not only to present the conclusions about the developed tool but also to share all conclusions associated with the prior theoretical study.

Furthermore, the potential future work that can be considered as a result of the research will be described.

\section{Final Result}
As a result of this project, a conversational tool has been created to support reminiscence therapy.

The tool consists of a chatbot capable of maintaining a conversation with the user, aiming to gather as much information as possible.

The tool can identify which information has not been obtained and generate coherent questions within the context of the conversation to gather the missing data. Additionally, it provides feedback that makes the chat a more meaningful conversation.

Images can be sent to the chatbot at any time, allowing it to provide a description and an associated question. This helps the user recall things more easily by answering more specific questions about an image they have in front of them.

The user interface is intuitive and easy to use. The fact that it is implemented through Telegram significantly reduces the learning curve and makes it much more usable for all types of users. Since it does not require any special technological knowledge, this tool is accessible not only to therapists but also to family members and patients who are familiar with mobile devices and computers.

Although initially designed as a therapeutic tool to gather information about the patient's life, this tool has multiple uses. For example, portable versions, such as a mobile or tablet application, allow it to be used as entertainment during long journeys, which can be enjoyable for users. Its ease of use also allows patients to use it during moments of loneliness, having interesting conversations about their life that stimulate their mind and provide virtual companionship.

The main function of this bot is to seek and store relevant information to generate life stories using other tools. Additionally, at the end of the conversation, a life story constructed with all the obtained information is generated. This result serves as feedback and a reward that keeps the user motivated to complete the conversation.

\section{Future Work}
As a final point of this work, the possible open lines of development and research that have arisen after analyzing the project and that could be studied to improve it are presented.

Firstly, although this work generates a small life story as the final result of the conversation, it would be interesting to connect this work with the project explained in section \ref{sec:trabajocristina} "Life Story Generation Using Deep Learning Techniques." For this, it would be beneficial to transform all the information obtained through the conversation with this chatbot into an input set in the appropriate format for the application developed in that work. This way, life stories more faithful to the person's real life would be generated, completing the full cycle of work prior to reminiscence therapy.

Additionally, another modification that could be interesting is adapting the model to multiple languages to expand the application's target audience. The Gemini API is available in different languages, but all predefined questions would need to be translated for user interaction.

The development of the Gemini API and access to new versions will bring new possibilities. Among them, the integration of videos to be sent to the chatbot for analysis to obtain more biographical information stands out. It might also be interesting to consider the alternative of voice recognition so that, in addition to the Telegram interface, a voice interface could be developed where the patient could speak and listen as if it were a phone call.

\end{otherlanguage}

% Bibliografía
%
% Si el TFM se escribe en inglés, editar TeXiS/TeXiS_bib para cambiar el
% estilo de las referencias
\include{Cascaras/bibliografia}
\makeBib

% Apéndices
\appendix
%\include{Apendices/appendixA}
%\include{Apendices/appendixB}
%\include{Apendices/appendixC}
%\include{...}
%\include{...}
%\include{...}
\backmatter



%
% Índice de palabras
%

% Sólo  la   generamos  si  está   declarada  \generaindice.  Consulta
% TeXiS.sty para más información.

% En realidad, el soporte para la generación de índices de palabras
% en TeXiS no está documentada en el manual, porque no ha sido usada
% "en producción". Por tanto, el fichero que genera el índice
% *no* se incluye aquí (está comentado). Consulta la documentación
% en TeXiS_pream.tex para más información.
\ifx\generaindice\undefined
\else
%\include{TeXiS/TeXiS_indice}
\fi

%
% Lista de acrónimos
%

% Sólo  lo  generamos  si  está declarada  \generaacronimos.  Consulta
% TeXiS.sty para más información.


\ifx\generaacronimos\undefined
\else
\include{TeXiS/TeXiS_acron}
\fi

%
% Final
%\include{Cascaras/fin}
%\end{otherlanguage}
\end{document}
