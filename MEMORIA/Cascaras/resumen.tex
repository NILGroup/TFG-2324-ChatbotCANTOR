\chapter*{Resumen}
\section*{\tituloPortadaVal}
La enfermedad de Alzheimer es una condición neurodegenerativa que afecta las funciones cognitivas, la memoria, el pensamiento y el comportamiento. Aunque no es reversible, existen distintos tratamientos para mejorar la calidad de vida de quienes la padecen.  Entre ellos destaca la terapia de reminiscencia. Este tratamiento no farmacológico se enfoca en evocar recuerdos haciendo que los pacientes ejerciten la mente y den un sentido de continuidad a su historia de vida.

Este proyecto tiene como objetivo desarrollar una herramienta que sirva de apoyo para la realización de estas terapias. En concreto, se centra en el desarrollo de un chatbot capaz de guiar el flujo de la conversación a lo largo de diferentes temas, extrayendo la información necesaria y generando preguntas especificas. 

Con este propósito la primera etapa del trabajo se centra en la realización de un estudio exhaustivo que permita conocer cuáles son las mejores herramientas para su implementación. En concreto, se estudiarán las principales APIs y LLMs de procesamiento del lenguaje.  A continuación, se desarrolla de forma iterativa e incremental el chatbot. 

La herramienta final se basa en la API de Gemini, especialmente en el modelo ``gemini-pro'' para implementar las funciones de análisis de respuestas y generación de preguntas. Por otro lado, la interfaz desarrollada a través de Telegram permite un uso fácil y eficiente desde distintos dispositivos (móviles, tablet u ordenador). 
 
El código asociado a este trabajo se puede consultar en el repositorio de \href{https://github.com/mavice07/TFG-ChatBotCantor.git}{GitHub.}

\section*{Palabras clave}
   
\noindent Terapia basada en reminiscencia, Chatbot, Historia de vida, Alzéhimer, API, Gemini, Procesamiento del lenguaje natural, LLM

   


