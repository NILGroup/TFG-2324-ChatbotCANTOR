\chapter*{Resumen}
\section*{\tituloPortadaVal}
La enfermedad de Alzheimer es una condición neurodegenerativa que afecta las funciones cognitivas, la memoria, el pensamiento y el comportamiento. Su llegada supone un cambio significativo en la vida de quienes la padecen y en su entorno. Actualmente, se estima que en España 900.000 personas sufren esta y otras formas de demencia, y se proyecta que los casos se duplicarán para el año 2050. Por lo tanto, es de vital importancia desarrollar técnicas que puedan ralentizar el avance de la enfermedad. Aunque no es reversible, existen terapias que pueden mejorar la calidad de vida tanto del paciente como de sus seres queridos.

La terapia de reminiscencia es una modalidad terapéutica que se enfoca en ayudar a las personas a recordar y compartir sus experiencias y recuerdos pasados, especialmente aquellos relacionados con eventos significativos en sus vidas. Aunque es comúnmente empleada con personas mayores, también puede resultar efectiva en otros grupos de edad.

Este proyecto se centra principalmente en el desarrollo de un ChatBot que interactúe con el paciente, recopilando la información necesaria para construir una historia de vida y las imágenes asociadas. Para lograrlo, se ha clonado la API de Gemini y se ha entrenado con datos personalizados, de manera que genere historias de vida específicas y pertinentes.

Este enfoque promete ser una herramienta valiosa para mejorar la calidad de vida de las personas afectadas por Alzheimer y demencias similares. Si necesitas más asistencia en relación a este proyecto, no dudes en decírmelo. 


\section*{Palabras clave}
   
\noindent Reminiscencia,Chatbot, historia de vida, alzéhimer

   


