\chapter*{Abstract}
\section*{\tituloPortadaVal}
Alzheimer's disease is a neurodegenerative condition that affects cognitive functions, memory, thinking, and behavior. Although it is not reversible, there are various treatments to improve the quality of life for those affected. Among them, reminiscence therapy stands out. This non-pharmacological treatment focuses on evoking memories, allowing patients to exercise their minds and give continuity to their life story.

This project aims to develop a tool to support the implementation of these therapies. Specifically, it focuses on developing a chatbot capable of guiding the conversation flow across different topics, extracting the necessary information, and generating specific questions.

To this end, the first stage of the work focuses on conducting an exhaustive study to determine the best tools for its implementation. Specifically, the main language processing APIs and LLMs will be studied. Subsequently, the chatbot is developed iteratively and incrementally.

The final tool is based on the Gemini API, specifically the "gemini-pro" model to implement response analysis and question generation functions. Additionally, the interface developed through Telegram allows for easy and efficient use from various devices (mobile, tablet, or computer).

The code associated with this work can be found in the \href{https://github.com/mavice07/TFG-ChatBotCantor.git}{GitHub repository.}

\section*{Keywords}

\noindent Reminiscence-based therapy, Chatbot, Life history, Alzheimer, API, Gemini, Natural language processing, LLM



