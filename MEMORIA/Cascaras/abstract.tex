\chapter*{Abstract}

\section*{\tituloPortadaEngVal}

Alzheimer's disease is a neurodegenerative condition that affects cognitive functions, memory, thinking, and behavior. Its onset marks a significant change in the lives of those affected and their surroundings. Currently, it is estimated that 900,000 people in Spain suffer from this and other forms of dementia, and it is projected that cases will double by the year 2050. Therefore, it is crucial to develop techniques that can slow down the progression of the disease. Although it is not reversible, there are therapies that can improve the quality of life for both the patient and their loved ones.

Reminiscence therapy is a therapeutic approach focused on helping individuals remember and share their past experiences and memories, especially those related to significant events in their lives. While commonly used with older individuals, it can also be effective with other age groups.

This project primarily focuses on the development of a ChatBot that interacts with the patient, gathering the necessary information to construct a life story along with associated images. To achieve this, the Gemini API has been cloned and trained with customized data, enabling it to generate specific and relevant life stories.

This approach holds promise as a valuable tool for enhancing the quality of life for individuals affected by Alzheimer's and similar dementias. 
\section*{Keywords}

\noindent reminiscence, chatbot, life story, Alzheimer's



