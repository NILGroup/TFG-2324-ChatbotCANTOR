\chapter*{Introduction}
\label{cap:introduction}
\addcontentsline{toc}{chapter}{Introduction}
\chapterquote{There are incurable patients, but none that are beyond care.}{Francesc Torralba}

In this first chapter, the reasons that have led me to undertake this project and the objectives sought to be achieved from the outset are presented.


\section{Motivation}
Currently, memory loss is a problem that affects a large portion of the population, from individuals with mild cognitive impairment to those suffering from more severe forms of dementia, such as Alzheimer's. This condition not only diminishes the patient's quality of life but also impacts the well-being of their family and loved ones. In Spain alone, according to the Spanish Society of Neurology, there are 800,000 people afflicted by this disease.

Non-pharmacological techniques have shown highly positive results in preserving memory, maintaining cognitive abilities, and retaining memories, thereby contributing to delaying cognitive decline. Among these, approaches based on reviewing the individual's own Life History have proven to be highly effective. This involves the patient, even one suffering from dementia, personally recording the most significant experiences, people, and places in their life. The aim is to stimulate the patient to discuss various topics, past events, and historical occurrences. It has been found that these exercises help preserve skills such as reasoning, self-esteem, confidence, and social abilities. Up until now, therapists implementing reminiscence therapies have manually created patients' life stories and relied on printed documents to prepare the sessions.

During the academic period of 2023-2024, the YayoBot application was developed with the purpose of facilitating therapists in conducting reminiscence-based therapies, streamlining and simplifying the process. This development starts from scratch and aims to create a functional and useful chatbot capable of engaging in a productive conversation with the patient to extract the necessary information.

This Bachelor's Thesis arises with the intention of assisting therapists, family members, or friends of dementia patients in obtaining the necessary material to apply reminiscence therapies on the patient. Thus, the goal is to enhance the effectiveness of these therapies and improve the quality of life for both the patients and their families.


\section{Goals}
This work aims to develop a chatbot from scratch using the Bard API that is capable of:
\begin{enumerate}

\item Developing an initial basic chatbot capable of responding to predefined questions and storing answers.
\item Enhancing the previous chatbot to make it more intelligent and capable of analyzing responses, identifying omitted information, asking specific questions to obtain missing information, etc.
\item Creating a final version of the chatbot that can analyze responses and engage in a meaningful conversation. It should also be able to generate appropriate questions. 
\end{enumerate}

\section{Work Plan}

The work plan to achieve the previously outlined objectives will be as follows:
\begin{enumerate}

\item Develop the first version of the chatbot by November 15th, with a focus on drafting the project introduction.

\item By the end of February, have the second version developed, along with 85% of the project report completed, primarily focusing on the core chapters.

\item Aim to complete the final version by May. Ensure that the code is finalized and the report includes all conclusions, results, and future work.

\end{enumerate}
To manage version control, we will use the GitHub repository at: https://github.com/NILGroup/TFG-2324-ChatbotCANTOR.












